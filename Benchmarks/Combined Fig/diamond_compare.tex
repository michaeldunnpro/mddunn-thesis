% Options for packages loaded elsewhere
% Options for packages loaded elsewhere
\PassOptionsToPackage{unicode}{hyperref}
\PassOptionsToPackage{hyphens}{url}
\PassOptionsToPackage{dvipsnames,svgnames,x11names}{xcolor}
%
\documentclass[
  letterpaper,
  DIV=11,
  numbers=noendperiod]{scrartcl}
\usepackage{xcolor}
\usepackage{amsmath,amssymb}
\setcounter{secnumdepth}{-\maxdimen} % remove section numbering
\usepackage{iftex}
\ifPDFTeX
  \usepackage[T1]{fontenc}
  \usepackage[utf8]{inputenc}
  \usepackage{textcomp} % provide euro and other symbols
\else % if luatex or xetex
  \usepackage{unicode-math} % this also loads fontspec
  \defaultfontfeatures{Scale=MatchLowercase}
  \defaultfontfeatures[\rmfamily]{Ligatures=TeX,Scale=1}
\fi
\usepackage{lmodern}
\ifPDFTeX\else
  % xetex/luatex font selection
\fi
% Use upquote if available, for straight quotes in verbatim environments
\IfFileExists{upquote.sty}{\usepackage{upquote}}{}
\IfFileExists{microtype.sty}{% use microtype if available
  \usepackage[]{microtype}
  \UseMicrotypeSet[protrusion]{basicmath} % disable protrusion for tt fonts
}{}
\makeatletter
\@ifundefined{KOMAClassName}{% if non-KOMA class
  \IfFileExists{parskip.sty}{%
    \usepackage{parskip}
  }{% else
    \setlength{\parindent}{0pt}
    \setlength{\parskip}{6pt plus 2pt minus 1pt}}
}{% if KOMA class
  \KOMAoptions{parskip=half}}
\makeatother
% Make \paragraph and \subparagraph free-standing
\makeatletter
\ifx\paragraph\undefined\else
  \let\oldparagraph\paragraph
  \renewcommand{\paragraph}{
    \@ifstar
      \xxxParagraphStar
      \xxxParagraphNoStar
  }
  \newcommand{\xxxParagraphStar}[1]{\oldparagraph*{#1}\mbox{}}
  \newcommand{\xxxParagraphNoStar}[1]{\oldparagraph{#1}\mbox{}}
\fi
\ifx\subparagraph\undefined\else
  \let\oldsubparagraph\subparagraph
  \renewcommand{\subparagraph}{
    \@ifstar
      \xxxSubParagraphStar
      \xxxSubParagraphNoStar
  }
  \newcommand{\xxxSubParagraphStar}[1]{\oldsubparagraph*{#1}\mbox{}}
  \newcommand{\xxxSubParagraphNoStar}[1]{\oldsubparagraph{#1}\mbox{}}
\fi
\makeatother

\usepackage{color}
\usepackage{fancyvrb}
\newcommand{\VerbBar}{|}
\newcommand{\VERB}{\Verb[commandchars=\\\{\}]}
\DefineVerbatimEnvironment{Highlighting}{Verbatim}{commandchars=\\\{\}}
% Add ',fontsize=\small' for more characters per line
\usepackage{framed}
\definecolor{shadecolor}{RGB}{241,243,245}
\newenvironment{Shaded}{\begin{snugshade}}{\end{snugshade}}
\newcommand{\AlertTok}[1]{\textcolor[rgb]{0.68,0.00,0.00}{#1}}
\newcommand{\AnnotationTok}[1]{\textcolor[rgb]{0.37,0.37,0.37}{#1}}
\newcommand{\AttributeTok}[1]{\textcolor[rgb]{0.40,0.45,0.13}{#1}}
\newcommand{\BaseNTok}[1]{\textcolor[rgb]{0.68,0.00,0.00}{#1}}
\newcommand{\BuiltInTok}[1]{\textcolor[rgb]{0.00,0.23,0.31}{#1}}
\newcommand{\CharTok}[1]{\textcolor[rgb]{0.13,0.47,0.30}{#1}}
\newcommand{\CommentTok}[1]{\textcolor[rgb]{0.37,0.37,0.37}{#1}}
\newcommand{\CommentVarTok}[1]{\textcolor[rgb]{0.37,0.37,0.37}{\textit{#1}}}
\newcommand{\ConstantTok}[1]{\textcolor[rgb]{0.56,0.35,0.01}{#1}}
\newcommand{\ControlFlowTok}[1]{\textcolor[rgb]{0.00,0.23,0.31}{\textbf{#1}}}
\newcommand{\DataTypeTok}[1]{\textcolor[rgb]{0.68,0.00,0.00}{#1}}
\newcommand{\DecValTok}[1]{\textcolor[rgb]{0.68,0.00,0.00}{#1}}
\newcommand{\DocumentationTok}[1]{\textcolor[rgb]{0.37,0.37,0.37}{\textit{#1}}}
\newcommand{\ErrorTok}[1]{\textcolor[rgb]{0.68,0.00,0.00}{#1}}
\newcommand{\ExtensionTok}[1]{\textcolor[rgb]{0.00,0.23,0.31}{#1}}
\newcommand{\FloatTok}[1]{\textcolor[rgb]{0.68,0.00,0.00}{#1}}
\newcommand{\FunctionTok}[1]{\textcolor[rgb]{0.28,0.35,0.67}{#1}}
\newcommand{\ImportTok}[1]{\textcolor[rgb]{0.00,0.46,0.62}{#1}}
\newcommand{\InformationTok}[1]{\textcolor[rgb]{0.37,0.37,0.37}{#1}}
\newcommand{\KeywordTok}[1]{\textcolor[rgb]{0.00,0.23,0.31}{\textbf{#1}}}
\newcommand{\NormalTok}[1]{\textcolor[rgb]{0.00,0.23,0.31}{#1}}
\newcommand{\OperatorTok}[1]{\textcolor[rgb]{0.37,0.37,0.37}{#1}}
\newcommand{\OtherTok}[1]{\textcolor[rgb]{0.00,0.23,0.31}{#1}}
\newcommand{\PreprocessorTok}[1]{\textcolor[rgb]{0.68,0.00,0.00}{#1}}
\newcommand{\RegionMarkerTok}[1]{\textcolor[rgb]{0.00,0.23,0.31}{#1}}
\newcommand{\SpecialCharTok}[1]{\textcolor[rgb]{0.37,0.37,0.37}{#1}}
\newcommand{\SpecialStringTok}[1]{\textcolor[rgb]{0.13,0.47,0.30}{#1}}
\newcommand{\StringTok}[1]{\textcolor[rgb]{0.13,0.47,0.30}{#1}}
\newcommand{\VariableTok}[1]{\textcolor[rgb]{0.07,0.07,0.07}{#1}}
\newcommand{\VerbatimStringTok}[1]{\textcolor[rgb]{0.13,0.47,0.30}{#1}}
\newcommand{\WarningTok}[1]{\textcolor[rgb]{0.37,0.37,0.37}{\textit{#1}}}

\usepackage{longtable,booktabs,array}
\usepackage{calc} % for calculating minipage widths
% Correct order of tables after \paragraph or \subparagraph
\usepackage{etoolbox}
\makeatletter
\patchcmd\longtable{\par}{\if@noskipsec\mbox{}\fi\par}{}{}
\makeatother
% Allow footnotes in longtable head/foot
\IfFileExists{footnotehyper.sty}{\usepackage{footnotehyper}}{\usepackage{footnote}}
\makesavenoteenv{longtable}
\usepackage{graphicx}
\makeatletter
\newsavebox\pandoc@box
\newcommand*\pandocbounded[1]{% scales image to fit in text height/width
  \sbox\pandoc@box{#1}%
  \Gscale@div\@tempa{\textheight}{\dimexpr\ht\pandoc@box+\dp\pandoc@box\relax}%
  \Gscale@div\@tempb{\linewidth}{\wd\pandoc@box}%
  \ifdim\@tempb\p@<\@tempa\p@\let\@tempa\@tempb\fi% select the smaller of both
  \ifdim\@tempa\p@<\p@\scalebox{\@tempa}{\usebox\pandoc@box}%
  \else\usebox{\pandoc@box}%
  \fi%
}
% Set default figure placement to htbp
\def\fps@figure{htbp}
\makeatother





\setlength{\emergencystretch}{3em} % prevent overfull lines

\providecommand{\tightlist}{%
  \setlength{\itemsep}{0pt}\setlength{\parskip}{0pt}}



 


\KOMAoption{captions}{tableheading}
\makeatletter
\@ifpackageloaded{caption}{}{\usepackage{caption}}
\AtBeginDocument{%
\ifdefined\contentsname
  \renewcommand*\contentsname{Table of contents}
\else
  \newcommand\contentsname{Table of contents}
\fi
\ifdefined\listfigurename
  \renewcommand*\listfigurename{List of Figures}
\else
  \newcommand\listfigurename{List of Figures}
\fi
\ifdefined\listtablename
  \renewcommand*\listtablename{List of Tables}
\else
  \newcommand\listtablename{List of Tables}
\fi
\ifdefined\figurename
  \renewcommand*\figurename{Figure}
\else
  \newcommand\figurename{Figure}
\fi
\ifdefined\tablename
  \renewcommand*\tablename{Table}
\else
  \newcommand\tablename{Table}
\fi
}
\@ifpackageloaded{float}{}{\usepackage{float}}
\floatstyle{ruled}
\@ifundefined{c@chapter}{\newfloat{codelisting}{h}{lop}}{\newfloat{codelisting}{h}{lop}[chapter]}
\floatname{codelisting}{Listing}
\newcommand*\listoflistings{\listof{codelisting}{List of Listings}}
\makeatother
\makeatletter
\makeatother
\makeatletter
\@ifpackageloaded{caption}{}{\usepackage{caption}}
\@ifpackageloaded{subcaption}{}{\usepackage{subcaption}}
\makeatother
\usepackage{bookmark}
\IfFileExists{xurl.sty}{\usepackage{xurl}}{} % add URL line breaks if available
\urlstyle{same}
\hypersetup{
  pdftitle={Simulation Comparison with Theory and Experiment for Light Ions in Diamond},
  pdfauthor={Michael Dunn},
  colorlinks=true,
  linkcolor={blue},
  filecolor={Maroon},
  citecolor={Blue},
  urlcolor={Blue},
  pdfcreator={LaTeX via pandoc}}


\title{Simulation Comparison with Theory and Experiment for Light Ions
in Diamond}
\author{Michael Dunn}
\date{2026-02-09}
\begin{document}
\maketitle


\section{Bethe Formalism}\label{bethe-formalism}

The Bethe stopping power formula, for nonrelativistic charged particles,
is typically given by:
\[-\frac{dE}{d \ell}  = \frac{4 \pi N Z^2}{m_e v^2}\left(\frac{q_e^2}{4 \pi \epsilon_0}\right)^2 \left[ \ln \left( \frac{2 m_e v^2}{I}\right)\right],\]
where \(q_e\) is the elementary charge, \(Z\) is the charge of the
incident particle, \(m_e\) is the electron mass, \(v\) is the velocity
of the incident particle, \(N\) is the number density of electrons in
the target material, and \(I\) is the mean excitation potential of the
target material. We consider \(\ell\) to be the path length of the
incident particle through the target material, and \(E\) to be the
kinetic energy of the incident particle. The stopping power is defined
to be precisely \(- dE/d \ell\), and is a measure of the energy loss of
the incident particle per unit path length through the target material.
This is a statistical consequence of quantum mechanics, and is therefore
an approximation of the energy loss of the incident particle;
nonetheless, we can treat stopping power classically and obtain strong
correspondence with actual results. Indeed, if one substitutes the
classical expression for the kinetic energy of a particle,
\[\frac{2 E}{m} = v^2,\] into the Bethe formula, one obtains:
\[-\frac{dE}{d \ell}  = \frac{4\pi {q_e}^4 Z^2}{m_e v^2} N \ln\left(\frac{4 m_e E}{I}\right)\]
which is a purely classical expression for the stopping power of a
charged particle in a target material. One can further simplify this by
defining a stopping power per \(Z^2\) as
\[S= -\frac{1}{Z^2}\frac{dE}{d \ell}\] and a reduced energy per ion unit
mass \[\varepsilon = \frac{E}{m},\] which removes dependence on the
incident particle's species. Doing so yields the function
\[S(\varepsilon) = \frac{2 \pi N}{\varepsilon m_e} \left(\frac{q_e^2}{4 \pi \epsilon_0}\right)^2 [\ln(4 \frac{m_e \varepsilon}{I})].\]
This will serve as a comparison point for our BCA simulation, along with
experimentally measured values.

\begin{Shaded}
\begin{Highlighting}[]
\ImportTok{import}\NormalTok{ numpy }\ImportTok{as}\NormalTok{ np}
\ImportTok{import}\NormalTok{ os}
\CommentTok{\# Relevant physical constants in SI base}
\NormalTok{epsilon\_0 }\OperatorTok{=} \FloatTok{8.854187817e{-}12} \CommentTok{\# Vacuum permittivity in F/m}
\NormalTok{q\_e }\OperatorTok{=} \FloatTok{1.602176634e{-}19} \CommentTok{\# Elementary charge in C}
\NormalTok{m\_e }\OperatorTok{=} \FloatTok{9.10938356e{-}31} \CommentTok{\# Electron mass in kg}
\NormalTok{N }\OperatorTok{=} \FloatTok{1.76e29}\OperatorTok{*}\DecValTok{6} \CommentTok{\# Number density of electrons in diamond in electrons/m\^{}3}
\CommentTok{\# Conversion factor from J to keV}
\NormalTok{J\_to\_keV }\OperatorTok{=} \FloatTok{6.242e15}
\NormalTok{m\_to\_microns }\OperatorTok{=} \FloatTok{1e6}
\NormalTok{amu\_to\_kg }\OperatorTok{=} \FloatTok{1.66053906660e{-}27}

\CommentTok{\# Ionization potential of diamond in eV}
\NormalTok{I }\OperatorTok{=} \FloatTok{81.0} \OperatorTok{*}\NormalTok{ q\_e }\CommentTok{\# Convert from eV to J}


\KeywordTok{def}\NormalTok{ S(epsilon):}
    \CommentTok{\# Convert epsilon from MeV/amu to J/kg}
\NormalTok{    epsilon }\OperatorTok{=}\NormalTok{ epsilon }\OperatorTok{*} \FloatTok{1e6} \OperatorTok{*}\NormalTok{ q\_e }\OperatorTok{/}\NormalTok{ amu\_to\_kg}
\NormalTok{    raw }\OperatorTok{=}\NormalTok{ (}\DecValTok{2} \OperatorTok{*}\NormalTok{ np.pi }\OperatorTok{*}\NormalTok{ N }\OperatorTok{/}\NormalTok{ (epsilon }\OperatorTok{*}\NormalTok{ m\_e) }\OperatorTok{*}\NormalTok{ (q\_e}\OperatorTok{**}\DecValTok{2} \OperatorTok{/}\NormalTok{ (}\DecValTok{4} \OperatorTok{*}\NormalTok{ np.pi }\OperatorTok{*}\NormalTok{ epsilon\_0))}\OperatorTok{**}\DecValTok{2} \OperatorTok{*}\NormalTok{ np.log(}\DecValTok{4} \OperatorTok{*}\NormalTok{ m\_e }\OperatorTok{*}\NormalTok{ epsilon }\OperatorTok{/}\NormalTok{ I))}
    \CommentTok{\# Convert from J/m to keV/micron}
    \ControlFlowTok{return}\NormalTok{ raw }\OperatorTok{*}\NormalTok{ J\_to\_keV }\OperatorTok{/}\NormalTok{ m\_to\_microns}
\end{Highlighting}
\end{Shaded}

\section{Simulation}\label{simulation}

Before we begin simulating, we need to point python to the location of
the RustBCA folder and a folder to output our results to. These paths
are set in the cell below, and should be updated to match your local
file structure.

\begin{Shaded}
\begin{Highlighting}[]
\CommentTok{\#\#\# Dependencies }\AlertTok{\#\#\#}
\CommentTok{\# RustBCA (Including compilation)}
\CommentTok{\# numpy, matplotlib}
\CommentTok{\# tomlkit}
\CommentTok{\# dask (for large energy loss files)}
\CommentTok{\# pyarrow, pandas (dependencies of dask)}
\CommentTok{\#\#\#{-}{-}{-}{-}{-}{-}{-}{-}{-}{-}{-}{-}{-}{-}}\AlertTok{\#\#\#}

\CommentTok{\#\#\#\#\#\#\#\#\#\#\#\#\#\#\#\#\#\#\#\#\#\#\#\#\#\#\#\#\#\#\#\#\#\#\#\#\#\#\#\#\#\#\#\#\#}
\CommentTok{\#\# RustBCA Directory (Absolute path)       \#\#}
\CommentTok{\#\#\#{-}{-}{-}{-}{-}{-}{-}{-}{-}{-}{-}{-}{-}{-}{-}{-}{-}{-}{-}{-}{-}{-}{-}{-}{-}{-}{-}{-}{-}{-}{-}{-}{-}{-}{-}{-}{-}{-}{-}}\AlertTok{\#\#\#}
\NormalTok{rustbca\_dir }\OperatorTok{=} \StringTok{"/Users/michaeldunn/Documents/Dev/2026/Thesis/RustBCA{-}Benchmarks/RustBCA"}
\CommentTok{\#\#\#{-}{-}{-}{-}{-}{-}{-}{-}{-}{-}{-}{-}{-}{-}{-}{-}{-}{-}{-}{-}{-}{-}{-}{-}{-}{-}{-}{-}{-}{-}{-}{-}{-}{-}{-}{-}{-}{-}{-}}\AlertTok{\#\#\#}
\CommentTok{\#\# End RustBCA Directory                   \#\#}
\CommentTok{\#\#\#\#\#\#\#\#\#\#\#\#\#\#\#\#\#\#\#\#\#\#\#\#\#\#\#\#\#\#\#\#\#\#\#\#\#\#\#\#\#\#\#\#\#}


\CommentTok{\#\#\#\#\#\#\#\#\#\#\#\#\#\#\#\#\#\#\#\#\#\#\#\#\#\#\#\#\#\#\#\#\#\#\#\#\#\#\#\#\#\#\#\#\#}
\CommentTok{\#\# Output Directory (Absolute path)        \#\#}
\CommentTok{\#\#\#{-}{-}{-}{-}{-}{-}{-}{-}{-}{-}{-}{-}{-}{-}{-}{-}{-}{-}{-}{-}{-}{-}{-}{-}{-}{-}{-}{-}{-}{-}{-}{-}{-}{-}{-}{-}{-}{-}{-}}\AlertTok{\#\#\#}
\NormalTok{output\_dir }\OperatorTok{=} \StringTok{"/Users/michaeldunn/Documents/Dev/2026/Thesis/RustBCA{-}Benchmarks/Benchmarks/Combined Fig/Data"}
\CommentTok{\#\#\#{-}{-}{-}{-}{-}{-}{-}{-}{-}{-}{-}{-}{-}{-}{-}{-}{-}{-}{-}{-}{-}{-}{-}{-}{-}{-}{-}{-}{-}{-}{-}{-}{-}{-}{-}{-}{-}{-}{-}}\AlertTok{\#\#\#}
\CommentTok{\#\# End Output Directory                    \#\#}
\CommentTok{\#\#\#\#\#\#\#\#\#\#\#\#\#\#\#\#\#\#\#\#\#\#\#\#\#\#\#\#\#\#\#\#\#\#\#\#\#\#\#\#\#\#\#\#\#}
\end{Highlighting}
\end{Shaded}

The following cells contain code for running simulations of helium,
hydrogen, and lithium atoms on diamond surfaces using binary collision
via RustBCA. The following collision geometry is used for all
simulations:

\begin{figure}[H]

{\centering \pandocbounded{\includegraphics[keepaspectratio]{diamond_compare_files/figure-pdf/3ac840b2-1-bca_geometry.png}}

}

\caption{bca\_geometry.png}

\end{figure}%

\begin{Shaded}
\begin{Highlighting}[]
\CommentTok{\#\#\# Dependencies }\AlertTok{\#\#\#}
\CommentTok{\# RustBCA}
\CommentTok{\# numpy, matplotlib}
\CommentTok{\# tomlkit}
\CommentTok{\# dask (for large energy loss files)}
\CommentTok{\# pyarrow, pandas (dependencies of dask)}
\CommentTok{\#\#\#{-}{-}{-}{-}{-}{-}{-}{-}{-}{-}{-}{-}{-}{-}}\AlertTok{\#\#\#}


\ImportTok{import}\NormalTok{ dask.dataframe }\ImportTok{as}\NormalTok{ dd}
\ImportTok{import}\NormalTok{ numpy }\ImportTok{as}\NormalTok{ np}
\ImportTok{import}\NormalTok{ pandas }\ImportTok{as}\NormalTok{ pd}
\ImportTok{import}\NormalTok{ matplotlib.pyplot }\ImportTok{as}\NormalTok{ plt}
\ImportTok{import}\NormalTok{ sys}
\ImportTok{import}\NormalTok{ os}
\ImportTok{from}\NormalTok{ libRustBCA }\ImportTok{import} \OperatorTok{*}



\NormalTok{os.chdir(rustbca\_dir)}
\CommentTok{\# Grab materials and formulas from scripts directory}
\NormalTok{sys.path.append(os.getcwd()}\OperatorTok{+}\StringTok{\textquotesingle{}/scripts\textquotesingle{}}\NormalTok{)}



\ImportTok{import}\NormalTok{ materials }\ImportTok{as}\NormalTok{ m}
\ImportTok{import}\NormalTok{ time}
\ImportTok{from}\NormalTok{ tomlkit }\ImportTok{import}\NormalTok{ dumps}

\CommentTok{\textquotesingle{}\textquotesingle{}\textquotesingle{}}
\CommentTok{This example simulates the implantation of various ions }
\CommentTok{at a 0 degree angle into a C diamond target of arbitrary thickness.}

\CommentTok{It creates an input file as a nested dictionary which is written to}
\CommentTok{a TOML file using tomlkit.}

\CommentTok{It runs the input file with cargo run {-}{-}release and reads the output files.}
\CommentTok{\textquotesingle{}\textquotesingle{}\textquotesingle{}}

\NormalTok{run\_sim }\OperatorTok{=} \VariableTok{True}
\NormalTok{mode }\OperatorTok{=} \StringTok{\textquotesingle{}0D\textquotesingle{}}
\NormalTok{number\_ions }\OperatorTok{=} \DecValTok{100} \CommentTok{\# higher energies allow smaller numbers of ions}
\NormalTok{angle }\OperatorTok{=} \FloatTok{0.1}   \CommentTok{\# degrees; measured from surface normal}

\CommentTok{\textquotesingle{}\textquotesingle{}\textquotesingle{}}
\CommentTok{For organizational purposes, species are commonly defined in dictionaries.}
\CommentTok{Additional examples can be found in scripts/materials.py, but values }
\CommentTok{should be checked for correctness before use. Values are explained}
\CommentTok{in the relevant sections below.}
\CommentTok{\textquotesingle{}\textquotesingle{}\textquotesingle{}}

\NormalTok{hydrogen }\OperatorTok{=}\NormalTok{ m.hydrogen}
\NormalTok{helium }\OperatorTok{=}\NormalTok{ m.helium}
\NormalTok{lithium }\OperatorTok{=}\NormalTok{ m.lithium}

\CommentTok{\# Gong et. all parameters for Diamond}
\NormalTok{diamond }\OperatorTok{=}\NormalTok{ \{}
    \StringTok{\textquotesingle{}symbol\textquotesingle{}}\NormalTok{: }\StringTok{\textquotesingle{}C\textquotesingle{}}\NormalTok{,}
    \StringTok{\textquotesingle{}name\textquotesingle{}}\NormalTok{: }\StringTok{\textquotesingle{}carbon\textquotesingle{}}\NormalTok{,}
    \StringTok{\textquotesingle{}Z\textquotesingle{}}\NormalTok{: }\FloatTok{6.0}\NormalTok{,}
    \StringTok{\textquotesingle{}m\textquotesingle{}}\NormalTok{: }\FloatTok{12.011}\NormalTok{, }\CommentTok{\# AMU }
    \StringTok{\textquotesingle{}Es\textquotesingle{}}\NormalTok{: }\FloatTok{7.41}\NormalTok{, }\CommentTok{\# eV}
    \StringTok{\textquotesingle{}Ec\textquotesingle{}}\NormalTok{: }\FloatTok{0.1}\NormalTok{, }\CommentTok{\# eV, reasonable for cutoff}
    \StringTok{\textquotesingle{}Eb\textquotesingle{}}\NormalTok{: }\FloatTok{7.36}\NormalTok{, }\CommentTok{\# eV}
    \StringTok{\textquotesingle{}Ed\textquotesingle{}}\NormalTok{: }\FloatTok{52.0}\NormalTok{, }\CommentTok{\# eV}
    \StringTok{\textquotesingle{}n\textquotesingle{}}\NormalTok{: }\FloatTok{1.76e29}\NormalTok{, }\CommentTok{\# 1/m\^{}3, atomic density of diamond}
\NormalTok{\}}

\NormalTok{options }\OperatorTok{=}\NormalTok{ \{}
    \StringTok{\textquotesingle{}name\textquotesingle{}}\NormalTok{: }\StringTok{\textquotesingle{}input\_file\textquotesingle{}}\NormalTok{,}
    \StringTok{\textquotesingle{}track\_trajectories\textquotesingle{}}\NormalTok{: }\VariableTok{False}\NormalTok{, }\CommentTok{\# whether to track trajectories for plotting; memory intensive}
    \StringTok{\textquotesingle{}track\_recoils\textquotesingle{}}\NormalTok{: }\VariableTok{False}\NormalTok{, }\CommentTok{\# whether to track recoils; must enable for sputtering}
    \StringTok{\textquotesingle{}track\_recoil\_trajectories\textquotesingle{}}\NormalTok{: }\VariableTok{False}\NormalTok{, }\CommentTok{\# whether to track recoil trajectories for plotting}
    \StringTok{\textquotesingle{}track\_displacements\textquotesingle{}}\NormalTok{: }\VariableTok{False}\NormalTok{, }\CommentTok{\# whether to track collisions with T \textgreater{} Ed for each species}
    \StringTok{\textquotesingle{}track\_energy\_losses\textquotesingle{}}\NormalTok{: }\VariableTok{True}\NormalTok{, }\CommentTok{\# whether to track detailed collision energies; memory intensive {-}{-} do not disable here.}
    \StringTok{\textquotesingle{}write\_buffer\_size\textquotesingle{}}\NormalTok{: }\DecValTok{2048}\NormalTok{, }\CommentTok{\# how big the buffer is for file writing}
    \StringTok{\textquotesingle{}weak\_collision\_order\textquotesingle{}}\NormalTok{: }\DecValTok{0}\NormalTok{, }\CommentTok{\# weak collisions at radii (k + 1)*r; enable only when required}
    \StringTok{\textquotesingle{}suppress\_deep\_recoils\textquotesingle{}}\NormalTok{: }\VariableTok{False}\NormalTok{, }\CommentTok{\# suppress recoils too deep to ever sputter}
    \StringTok{\textquotesingle{}high\_energy\_free\_flight\_paths\textquotesingle{}}\NormalTok{: }\VariableTok{False}\NormalTok{, }\CommentTok{\# SRIM{-}style high energy free flight distances; use with caution}
    \StringTok{\textquotesingle{}num\_threads\textquotesingle{}}\NormalTok{: os.cpu\_count(), }\CommentTok{\# number of threads to run in parallel}
    \StringTok{\textquotesingle{}num\_chunks\textquotesingle{}}\NormalTok{: }\DecValTok{100}\NormalTok{, }\CommentTok{\# code will write to file every nth chunk; for very large simulations, increase num\_chunks}
    \StringTok{\textquotesingle{}electronic\_stopping\_mode\textquotesingle{}}\NormalTok{: }\StringTok{\textquotesingle{}INTERPOLATED\textquotesingle{}}\NormalTok{, }\CommentTok{\# Previously \textquotesingle{}LOW\_ENERGY\_NONLOCAL\textquotesingle{}, leads to order of magnitude errors. Use \textquotesingle{}INTERPOLATED\textquotesingle{} instead.}
    \StringTok{\textquotesingle{}mean\_free\_path\_model\textquotesingle{}}\NormalTok{: }\StringTok{\textquotesingle{}LIQUID\textquotesingle{}}\NormalTok{, }\CommentTok{\# liquid is amorphous (constant mean free path); gas is exponentially{-}distributed mean free paths}
    \StringTok{\textquotesingle{}interaction\_potential\textquotesingle{}}\NormalTok{: [[}\StringTok{\textquotesingle{}ZBL\textquotesingle{}}\NormalTok{]], }\CommentTok{\# ZBL potential chosen for all interactions}
    \StringTok{\textquotesingle{}scattering\_integral\textquotesingle{}}\NormalTok{: [}
\NormalTok{        [}
\NormalTok{            \{}
                \StringTok{\textquotesingle{}GAUSS\_MEHLER\textquotesingle{}}\NormalTok{: \{}\StringTok{\textquotesingle{}n\_points\textquotesingle{}}\NormalTok{: }\DecValTok{6}\NormalTok{\}}
\NormalTok{            \}}
\NormalTok{        ]}
\NormalTok{    ],}

    \StringTok{\textquotesingle{}root\_finder\textquotesingle{}}\NormalTok{: [}
\NormalTok{        [}
\NormalTok{            \{}
                \StringTok{\textquotesingle{}NEWTON\textquotesingle{}}\NormalTok{: \{}
                    \StringTok{\textquotesingle{}max\_iterations\textquotesingle{}}\NormalTok{: }\DecValTok{100}\NormalTok{,}
                    \StringTok{\textquotesingle{}tolerance\textquotesingle{}}\NormalTok{: }\FloatTok{1e{-}6}
\NormalTok{                \}}
\NormalTok{            \}}
\NormalTok{        ]}
\NormalTok{    ],}
\NormalTok{\}}

\NormalTok{geometry\_0D }\OperatorTok{=}\NormalTok{ \{}
    \StringTok{\textquotesingle{}length\_unit\textquotesingle{}}\NormalTok{: }\StringTok{\textquotesingle{}ANGSTROM\textquotesingle{}}\NormalTok{,}
    \CommentTok{\# used to correct nonlocal stopping for known compound discrpancies}
    \StringTok{\textquotesingle{}electronic\_stopping\_correction\_factor\textquotesingle{}}\NormalTok{: }\FloatTok{0.0}\NormalTok{,}
    \CommentTok{\# number densities of each species}
    \StringTok{\textquotesingle{}densities\textquotesingle{}}\NormalTok{: [diamond[}\StringTok{"n"}\NormalTok{] }\OperatorTok{/} \FloatTok{1e30}\NormalTok{]}
\NormalTok{\}}

\NormalTok{material\_parameters }\OperatorTok{=}\NormalTok{ \{}
    \StringTok{\textquotesingle{}energy\_unit\textquotesingle{}}\NormalTok{: }\StringTok{\textquotesingle{}EV\textquotesingle{}}\NormalTok{,}
    \StringTok{\textquotesingle{}mass\_unit\textquotesingle{}}\NormalTok{: }\StringTok{\textquotesingle{}AMU\textquotesingle{}}\NormalTok{,}
    \CommentTok{\# bulk binding energy; typically zero as a model choice}
    \StringTok{\textquotesingle{}Eb\textquotesingle{}}\NormalTok{: [}
\NormalTok{        diamond[}\StringTok{"Eb"}\NormalTok{],}
\NormalTok{    ],}
    \CommentTok{\# surface binding energy}
    \StringTok{\textquotesingle{}Es\textquotesingle{}}\NormalTok{: [}
\NormalTok{        diamond[}\StringTok{"Es"}\NormalTok{],    }
\NormalTok{        ],}
    \CommentTok{\# cutoff energy {-} particles with E \textless{} Ec stop}
    \StringTok{\textquotesingle{}Ec\textquotesingle{}}\NormalTok{: [}
\NormalTok{        diamond[}\StringTok{"Ec"}\NormalTok{]}
\NormalTok{    ],}
    \CommentTok{\# displacement energy {-} only used to track displacements}
    \StringTok{\textquotesingle{}Ed\textquotesingle{}}\NormalTok{: [}
\NormalTok{        diamond[}\StringTok{"Ed"}\NormalTok{]}
\NormalTok{        ],}
    \CommentTok{\# atomic number}
    \StringTok{\textquotesingle{}Z\textquotesingle{}}\NormalTok{: [}
\NormalTok{        diamond[}\StringTok{"Z"}\NormalTok{]}
\NormalTok{    ],}
    \CommentTok{\# atomic mass}
    \StringTok{\textquotesingle{}m\textquotesingle{}}\NormalTok{: [}
\NormalTok{        diamond[}\StringTok{"m"}\NormalTok{]}
\NormalTok{    ],}
    \CommentTok{\# used to pick interaction potential from matrix in [options]}
    \StringTok{\textquotesingle{}interaction\_index\textquotesingle{}}\NormalTok{: [}\DecValTok{0}\NormalTok{, }\DecValTok{0}\NormalTok{],}
    \StringTok{\textquotesingle{}surface\_binding\_model\textquotesingle{}}\NormalTok{: \{}
        \StringTok{"PLANAR"}\NormalTok{: \{}\StringTok{\textquotesingle{}calculation\textquotesingle{}}\NormalTok{: }\StringTok{"INDIVIDUAL"}\NormalTok{\}}
\NormalTok{    \},}
    \StringTok{\textquotesingle{}bulk\_binding\_model\textquotesingle{}}\NormalTok{: }\StringTok{\textquotesingle{}INDIVIDUAL\textquotesingle{}}
\NormalTok{\}}
\end{Highlighting}
\end{Shaded}

\subsection{Hydrogen}\label{hydrogen}

\begin{Shaded}
\begin{Highlighting}[]
\CommentTok{\#\#\#\#\#\#\#\#\#\#\#\#\#\#\#\#\#\#\#\#\#\#\#\#\#\#\#\#\#\#\#\#\#\#\#\#\#\#\#\#\#\#\#\#\#}
\CommentTok{\#\# Simulation Options                      \#\#}
\CommentTok{\#\#\#{-}{-}{-}{-}{-}{-}{-}{-}{-}{-}{-}{-}{-}{-}{-}{-}{-}{-}{-}{-}{-}{-}{-}{-}{-}{-}{-}{-}{-}{-}{-}{-}{-}{-}{-}{-}{-}{-}{-}}\AlertTok{\#\#\#}
 \CommentTok{\# Hydrogen ion energies to simulate in MeV/amu}
\NormalTok{hydrogen\_energies }\OperatorTok{=}\NormalTok{ np.arange(}\FloatTok{1.5}\NormalTok{, }\FloatTok{6.1}\NormalTok{, }\FloatTok{0.5}\NormalTok{)}
 \CommentTok{\# Read range of energy deposition in Angstroms, and number of bins (e.g., numerical subdivisions of integral)}
\NormalTok{energy\_read\_range }\OperatorTok{=}\NormalTok{ np.linspace(}\FloatTok{0.0}\NormalTok{, }\FloatTok{3e5}\NormalTok{, }\DecValTok{10001}\NormalTok{)}
\CommentTok{\#\#\#{-}{-}{-}{-}{-}{-}{-}{-}{-}{-}{-}{-}{-}{-}{-}{-}{-}{-}{-}{-}{-}{-}{-}{-}{-}{-}{-}{-}{-}{-}{-}{-}{-}{-}{-}{-}{-}{-}{-}}\AlertTok{\#\#\#}
\CommentTok{\#\# End Simulation Options                  \#\#}
\CommentTok{\#\#\#\#\#\#\#\#\#\#\#\#\#\#\#\#\#\#\#\#\#\#\#\#\#\#\#\#\#\#\#\#\#\#\#\#\#\#\#\#\#\#\#\#\#}


\CommentTok{\# Convert hydrogen energies to raw energies in eV for input file}
\NormalTok{hydrogen\_energies\_in }\OperatorTok{=}\NormalTok{ hydrogen\_energies }\OperatorTok{*} \FloatTok{1e6} \OperatorTok{*}\NormalTok{ hydrogen[}\StringTok{"m"}\NormalTok{]}

\NormalTok{os.chdir(rustbca\_dir)}
\NormalTok{stopping\_data }\OperatorTok{=}\NormalTok{ [] }\CommentTok{\# Collect stopping powers here in list}

\NormalTok{particle\_parameters }\OperatorTok{=}\NormalTok{ \{}
    \StringTok{\textquotesingle{}length\_unit\textquotesingle{}}\NormalTok{: }\StringTok{\textquotesingle{}ANGSTROM\textquotesingle{}}\NormalTok{,}
    \StringTok{\textquotesingle{}energy\_unit\textquotesingle{}}\NormalTok{: }\StringTok{\textquotesingle{}EV\textquotesingle{}}\NormalTok{,}
    \StringTok{\textquotesingle{}mass\_unit\textquotesingle{}}\NormalTok{: }\StringTok{\textquotesingle{}AMU\textquotesingle{}}\NormalTok{,}
    \CommentTok{\# number of computational ions of this species to run at this energy}
    \StringTok{\textquotesingle{}N\textquotesingle{}}\NormalTok{: [number\_ions],}
    \CommentTok{\# atomic mass}
    \StringTok{\textquotesingle{}m\textquotesingle{}}\NormalTok{: [hydrogen[}\StringTok{"m"}\NormalTok{]],}
    \CommentTok{\# atomic number}
    \StringTok{\textquotesingle{}Z\textquotesingle{}}\NormalTok{: [hydrogen[}\StringTok{"Z"}\NormalTok{]],}
    \CommentTok{\# incidenet energy }
    \StringTok{\textquotesingle{}E\textquotesingle{}}\NormalTok{: [}\FloatTok{0.0}\NormalTok{], }\CommentTok{\# Changed in loop}
    \CommentTok{\# cutoff energy {-} if E \textless{} Ec, particle stops}
    \StringTok{\textquotesingle{}Ec\textquotesingle{}}\NormalTok{: [hydrogen[}\StringTok{"Ec"}\NormalTok{]],}
    \CommentTok{\# surface binding energy}
    \StringTok{\textquotesingle{}Es\textquotesingle{}}\NormalTok{: [hydrogen[}\StringTok{"Es"}\NormalTok{]],}
    \CommentTok{\# initial position {-} if Es significant and E low, start (n)\^{}({-}1/3) above surface}
    \CommentTok{\# otherwise 0, 0, 0 is fine; most geometry modes have surface at x=0 with target x\textgreater{}0}
    \StringTok{\textquotesingle{}pos\textquotesingle{}}\NormalTok{: [[}\FloatTok{0.0}\NormalTok{, }\FloatTok{0.0}\NormalTok{, }\FloatTok{0.0}\NormalTok{]],}
    \CommentTok{\# initial direction unit vector; most geometry modes have x{-}axis into the surface}
    \StringTok{\textquotesingle{}dir\textquotesingle{}}\NormalTok{: [}
\NormalTok{        [}
\NormalTok{            np.cos(angle}\OperatorTok{*}\NormalTok{np.pi}\OperatorTok{/}\FloatTok{180.0}\NormalTok{),}
\NormalTok{            np.sin(angle}\OperatorTok{*}\NormalTok{np.pi}\OperatorTok{/}\FloatTok{180.0}\NormalTok{),}
            \FloatTok{0.0}
\NormalTok{        ]}
\NormalTok{        ],}
\NormalTok{    \}}

\CommentTok{\# Relevant constants}

\NormalTok{Z }\OperatorTok{=}\NormalTok{ hydrogen[}\StringTok{"Z"}\NormalTok{]}
\NormalTok{M }\OperatorTok{=}\NormalTok{ hydrogen[}\StringTok{"m"}\NormalTok{]}


\CommentTok{\# Loop over incident energies}
\ControlFlowTok{for}\NormalTok{ incident\_energy\_mev\_per\_amu, incident\_energy }\KeywordTok{in} \BuiltInTok{zip}\NormalTok{(hydrogen\_energies, hydrogen\_energies\_in):}
    \BuiltInTok{print}\NormalTok{(}\SpecialStringTok{f\textquotesingle{}Running simulation for incident energy: }\SpecialCharTok{\{}\NormalTok{incident\_energy\_mev\_per\_amu}\SpecialCharTok{\}}\SpecialStringTok{ MeV/amu\textquotesingle{}}\NormalTok{)}
\NormalTok{    particle\_parameters[}\StringTok{\textquotesingle{}E\textquotesingle{}}\NormalTok{] }\OperatorTok{=}\NormalTok{ [incident\_energy]}
\NormalTok{    input\_data }\OperatorTok{=}\NormalTok{ \{}
        \StringTok{\textquotesingle{}options\textquotesingle{}}\NormalTok{: options,}
        \StringTok{\textquotesingle{}material\_parameters\textquotesingle{}}\NormalTok{: material\_parameters,}
        \StringTok{\textquotesingle{}particle\_parameters\textquotesingle{}}\NormalTok{: particle\_parameters,}
        \StringTok{\textquotesingle{}geometry\_input\textquotesingle{}}\NormalTok{: geometry\_0D}
\NormalTok{    \}}

    \CommentTok{\# Attempt to cleanup line endings}
\NormalTok{    input\_string }\OperatorTok{=}\NormalTok{ dumps(input\_data).replace(}\StringTok{\textquotesingle{}}\CharTok{\textbackslash{}r}\StringTok{\textquotesingle{}}\NormalTok{, }\StringTok{\textquotesingle{}\textquotesingle{}}\NormalTok{)}
    \ControlFlowTok{with}  \BuiltInTok{open}\NormalTok{(}\StringTok{\textquotesingle{}examples/input\_file.toml\textquotesingle{}}\NormalTok{, }\StringTok{\textquotesingle{}w\textquotesingle{}}\NormalTok{) }\ImportTok{as}\NormalTok{ input\_file:}
\NormalTok{        input\_file.write(input\_string)}

    \ControlFlowTok{if}\NormalTok{ run\_sim:}
\NormalTok{        os.system(}\SpecialStringTok{f\textquotesingle{}cargo run {-}{-}release }\SpecialCharTok{\{}\NormalTok{mode}\SpecialCharTok{\}}\SpecialStringTok{ examples/input\_file.toml\textquotesingle{}}\NormalTok{)}

    \CommentTok{\# Read CSV in chunks to avoid memory issues}
\NormalTok{    loss\_df }\OperatorTok{=}\NormalTok{ dd.read\_csv(}\StringTok{\textquotesingle{}input\_fileenergy\_loss.output\textquotesingle{}}\NormalTok{,}
\NormalTok{                    header}\OperatorTok{=}\VariableTok{None}\NormalTok{,}
\NormalTok{                    dtype}\OperatorTok{=} \BuiltInTok{float}\NormalTok{,}
\NormalTok{                    blocksize}\OperatorTok{=}\StringTok{"64MB"}\NormalTok{).dropna()}
    
    \CommentTok{\# Process histogram in chunks without loading all data into memory}
\NormalTok{    depth\_col }\OperatorTok{=} \DecValTok{4}
\NormalTok{    energy\_cols }\OperatorTok{=}\NormalTok{ [}\DecValTok{2}\NormalTok{, }\DecValTok{3}\NormalTok{]}
    
    \CommentTok{\# Construct histogram for energy deposition function to measurement set by user}
\NormalTok{    bin\_edges }\OperatorTok{=}\NormalTok{ energy\_read\_range}
\NormalTok{    hist }\OperatorTok{=}\NormalTok{ np.zeros(}\BuiltInTok{len}\NormalTok{(bin\_edges) }\OperatorTok{{-}} \DecValTok{1}\NormalTok{)}
    
    \CommentTok{\# Process in chunks}
    \ControlFlowTok{for}\NormalTok{ partition }\KeywordTok{in}\NormalTok{ loss\_df.to\_delayed():}
\NormalTok{        chunk }\OperatorTok{=}\NormalTok{ partition.compute()}
        \ControlFlowTok{if} \BuiltInTok{len}\NormalTok{(chunk) }\OperatorTok{\textgreater{}} \DecValTok{0}\NormalTok{:}
\NormalTok{            depth }\OperatorTok{=}\NormalTok{ chunk.iloc[:, depth\_col].values}
\NormalTok{            energy }\OperatorTok{=}\NormalTok{ chunk.iloc[:, energy\_cols[}\DecValTok{0}\NormalTok{]].values }\OperatorTok{+}\NormalTok{ chunk.iloc[:, energy\_cols[}\DecValTok{1}\NormalTok{]].values}
\NormalTok{            chunk\_hist, \_ }\OperatorTok{=}\NormalTok{ np.histogram(depth, bins}\OperatorTok{=}\NormalTok{bin\_edges, weights}\OperatorTok{=}\NormalTok{energy)}
\NormalTok{            hist }\OperatorTok{+=}\NormalTok{ chunk\_hist}
    
\NormalTok{    loss }\OperatorTok{=} \VariableTok{None}  \CommentTok{\# Don\textquotesingle{}t need full array anymore}

    \ControlFlowTok{if}\NormalTok{ hist.}\BuiltInTok{sum}\NormalTok{() }\OperatorTok{==} \DecValTok{0}\NormalTok{:}
        \BuiltInTok{print}\NormalTok{(}\StringTok{\textquotesingle{}No energy loss data\textquotesingle{}}\NormalTok{)}
    \ControlFlowTok{else}\NormalTok{:}
        \CommentTok{\# Histogram energy loss data already computed above}
        \CommentTok{\# Total x coordinate range is 0 to 300000 A with 10000 bins}
\NormalTok{        widths }\OperatorTok{=}\NormalTok{ np.diff(bin\_edges)}
\NormalTok{        energy\_density }\OperatorTok{=}\NormalTok{ np.divide(}
\NormalTok{            hist,}
\NormalTok{            widths }\OperatorTok{*}\NormalTok{ number\_ions,}
\NormalTok{            out}\OperatorTok{=}\NormalTok{np.zeros\_like(hist, dtype}\OperatorTok{=}\BuiltInTok{float}\NormalTok{),}
\NormalTok{            where}\OperatorTok{=}\NormalTok{widths }\OperatorTok{!=} \DecValTok{0}
\NormalTok{        )}

\NormalTok{    total\_loss\_per\_ion }\OperatorTok{=}\NormalTok{ hist.}\BuiltInTok{sum}\NormalTok{() }\OperatorTok{/}\NormalTok{ number\_ions}

    \BuiltInTok{print}\NormalTok{(}\StringTok{\textquotesingle{}Total energy loss per ion:\textquotesingle{}}\NormalTok{, total\_loss\_per\_ion, }\StringTok{\textquotesingle{}eV\textquotesingle{}}\NormalTok{)}

    \CommentTok{\# Print energy per ion in particular range (e.g., 0 to 1000 A)}
\NormalTok{    range\_min }\OperatorTok{=} \FloatTok{0.0} \CommentTok{\# Depletion region}
\NormalTok{    range\_max }\OperatorTok{=} \FloatTok{35000.0} \CommentTok{\#  End of target}
\NormalTok{    mask }\OperatorTok{=}\NormalTok{ (bin\_edges[:}\OperatorTok{{-}}\DecValTok{1}\NormalTok{] }\OperatorTok{\textgreater{}=}\NormalTok{ range\_min) }\OperatorTok{\&}\NormalTok{ (bin\_edges[:}\OperatorTok{{-}}\DecValTok{1}\NormalTok{] }\OperatorTok{\textless{}}\NormalTok{ range\_max)}
\NormalTok{    energy\_in\_range }\OperatorTok{=}\NormalTok{ np.}\BuiltInTok{sum}\NormalTok{(energy\_density[mask] }\OperatorTok{*}\NormalTok{ widths[mask])}
\NormalTok{    percent\_loss\_in\_range }\OperatorTok{=}\NormalTok{ (energy\_in\_range }\OperatorTok{/}\NormalTok{ incident\_energy) }\OperatorTok{*} \FloatTok{100.0}
\NormalTok{    stopping\_power }\OperatorTok{=}\NormalTok{ energy\_in\_range }\OperatorTok{/}\NormalTok{ (range\_max }\OperatorTok{{-}}\NormalTok{ range\_min)}
    \BuiltInTok{print}\NormalTok{(}\SpecialStringTok{f\textquotesingle{}Energy loss per ion in range }\SpecialCharTok{\{}\NormalTok{range\_min}\SpecialCharTok{\}}\SpecialStringTok{ A to }\SpecialCharTok{\{}\NormalTok{range\_max}\SpecialCharTok{\}}\SpecialStringTok{ A: }\SpecialCharTok{\{}\NormalTok{energy\_in\_range}\SpecialCharTok{\}}\SpecialStringTok{ eV (}\SpecialCharTok{\{}\NormalTok{percent\_loss\_in\_range}\SpecialCharTok{:.2f\}}\SpecialStringTok{\%)\textquotesingle{}}\NormalTok{)}
    \BuiltInTok{print}\NormalTok{(}\SpecialStringTok{f\textquotesingle{}Stopping power in range }\SpecialCharTok{\{}\NormalTok{range\_min}\SpecialCharTok{\}}\SpecialStringTok{ A to }\SpecialCharTok{\{}\NormalTok{range\_max}\SpecialCharTok{\}}\SpecialStringTok{ A: }\SpecialCharTok{\{}\NormalTok{stopping\_power}\SpecialCharTok{\}}\SpecialStringTok{ eV/A/ion\textquotesingle{}}\NormalTok{)}
    \CommentTok{\# Convert to stopping power in KeV/um/Z**2 for comparison}
\NormalTok{    stopping\_power\_keV\_per\_um\_per\_Z2 }\OperatorTok{=}\NormalTok{ (stopping\_power }\OperatorTok{*} \FloatTok{1e{-}3}\NormalTok{) }\OperatorTok{/}\NormalTok{ (}\FloatTok{1e{-}4}\NormalTok{) }\OperatorTok{/}\NormalTok{ (Z}\OperatorTok{**}\DecValTok{2}\NormalTok{)}
    \CommentTok{\# Convert energy to MeV/amu for comparison}
\NormalTok{    incident\_energy\_MeV\_per\_amu }\OperatorTok{=}\NormalTok{ incident\_energy\_mev\_per\_amu}
\NormalTok{    stopping\_data.append(\{}
        \StringTok{\textquotesingle{}Incident Energy (MeV/amu)\textquotesingle{}}\NormalTok{: incident\_energy\_MeV\_per\_amu,}
        \StringTok{\textquotesingle{}Stopping Power (KeV/um/Z\^{}2)\textquotesingle{}}\NormalTok{: stopping\_power\_keV\_per\_um\_per\_Z2,}
        \StringTok{\textquotesingle{}Percent Energy Loss (\%)\textquotesingle{}}\NormalTok{: percent\_loss\_in\_range}
\NormalTok{    \})}

\CommentTok{\# Revert to script directory for output}
\NormalTok{os.chdir(output\_dir)}
\CommentTok{\# Write to file}
\NormalTok{stopping\_powers\_df }\OperatorTok{=}\NormalTok{ pd.DataFrame(stopping\_data)}
\NormalTok{stopping\_powers\_df.to\_csv(}\StringTok{\textquotesingle{}diamond\_h\_stopping\_powers.csv\textquotesingle{}}\NormalTok{, index}\OperatorTok{=}\VariableTok{False}\NormalTok{)}
\end{Highlighting}
\end{Shaded}

\begin{verbatim}
Running simulation for incident energy: 1.5 MeV/amu
\end{verbatim}

\begin{verbatim}
    Finished `release` profile [optimized] target(s) in 0.07s
     Running `target/release/RustBCA 0D examples/input_file.toml`
\end{verbatim}

\begin{verbatim}
Processing 100 ions...
Initializing with 10 threads...
Finished!
Total energy loss per ion: 1529677.6647846296 eV
Energy loss per ion in range 0.0 A to 35000.0 A: 233895.41948476707 eV (15.47%)
Stopping power in range 0.0 A to 35000.0 A: 6.682726270993345 eV/A/ion
Running simulation for incident energy: 2.0 MeV/amu
\end{verbatim}

\begin{verbatim}
    Finished `release` profile [optimized] target(s) in 0.12s
     Running `target/release/RustBCA 0D examples/input_file.toml`
\end{verbatim}

\begin{verbatim}
Processing 100 ions...
Initializing with 10 threads...
Finished!
Total energy loss per ion: 2041159.0404275106 eV
Energy loss per ion in range 0.0 A to 35000.0 A: 184097.76179667868 eV (9.13%)
Stopping power in range 0.0 A to 35000.0 A: 5.259936051333677 eV/A/ion
Running simulation for incident energy: 2.5 MeV/amu
\end{verbatim}

\begin{verbatim}
    Finished `release` profile [optimized] target(s) in 0.11s
     Running `target/release/RustBCA 0D examples/input_file.toml`
\end{verbatim}

\begin{verbatim}
Processing 100 ions...
Initializing with 10 threads...
Finished!
\end{verbatim}

\subsection{Helium}\label{helium}

\begin{Shaded}
\begin{Highlighting}[]
\CommentTok{\#\#\#\#\#\#\#\#\#\#\#\#\#\#\#\#\#\#\#\#\#\#\#\#\#\#\#\#\#\#\#\#\#\#\#\#\#\#\#\#\#\#\#\#\#}
\CommentTok{\#\# Simulation Options                      \#\#}
\CommentTok{\#\#\#{-}{-}{-}{-}{-}{-}{-}{-}{-}{-}{-}{-}{-}{-}{-}{-}{-}{-}{-}{-}{-}{-}{-}{-}{-}{-}{-}{-}{-}{-}{-}{-}{-}{-}{-}{-}{-}{-}{-}}\AlertTok{\#\#\#}
 \CommentTok{\# Helium ion energies to simulate in MeV/amu}
\NormalTok{helium\_energies }\OperatorTok{=}\NormalTok{ np.arange(}\FloatTok{10.0}\NormalTok{, }\FloatTok{30.0}\NormalTok{, }\FloatTok{1.0}\NormalTok{)}
 \CommentTok{\# Read range of energy deposition in Angstroms, and number of bins (e.g., numerical subdivisions of integral)}
\NormalTok{energy\_read\_range }\OperatorTok{=}\NormalTok{ np.linspace(}\FloatTok{0.0}\NormalTok{, }\FloatTok{3e5}\NormalTok{, }\DecValTok{10001}\NormalTok{)}
\CommentTok{\#\#\#{-}{-}{-}{-}{-}{-}{-}{-}{-}{-}{-}{-}{-}{-}{-}{-}{-}{-}{-}{-}{-}{-}{-}{-}{-}{-}{-}{-}{-}{-}{-}{-}{-}{-}{-}{-}{-}{-}{-}}\AlertTok{\#\#\#}
\CommentTok{\#\# End Simulation Options                  \#\#}
\CommentTok{\#\#\#\#\#\#\#\#\#\#\#\#\#\#\#\#\#\#\#\#\#\#\#\#\#\#\#\#\#\#\#\#\#\#\#\#\#\#\#\#\#\#\#\#\#}

\CommentTok{\# Convert helium energies to raw energies in eV for input file}
\NormalTok{helium\_energies\_in }\OperatorTok{=}\NormalTok{ helium\_energies }\OperatorTok{*} \FloatTok{1e6} \OperatorTok{*}\NormalTok{ helium[}\StringTok{"m"}\NormalTok{]}

\NormalTok{os.chdir(rustbca\_dir)}
\NormalTok{stopping\_data }\OperatorTok{=}\NormalTok{ [] }\CommentTok{\# Collect stopping powers here in list}

\NormalTok{helium }\OperatorTok{=}\NormalTok{ m.helium}

\NormalTok{particle\_parameters }\OperatorTok{=}\NormalTok{ \{}
    \StringTok{\textquotesingle{}length\_unit\textquotesingle{}}\NormalTok{: }\StringTok{\textquotesingle{}ANGSTROM\textquotesingle{}}\NormalTok{,}
    \StringTok{\textquotesingle{}energy\_unit\textquotesingle{}}\NormalTok{: }\StringTok{\textquotesingle{}EV\textquotesingle{}}\NormalTok{,}
    \StringTok{\textquotesingle{}mass\_unit\textquotesingle{}}\NormalTok{: }\StringTok{\textquotesingle{}AMU\textquotesingle{}}\NormalTok{,}
    \CommentTok{\# number of computational ions of this species to run at this energy}
    \StringTok{\textquotesingle{}N\textquotesingle{}}\NormalTok{: [number\_ions],}
    \CommentTok{\# atomic mass}
    \StringTok{\textquotesingle{}m\textquotesingle{}}\NormalTok{: [helium[}\StringTok{"m"}\NormalTok{]],}
    \CommentTok{\# atomic number}
    \StringTok{\textquotesingle{}Z\textquotesingle{}}\NormalTok{: [helium[}\StringTok{"Z"}\NormalTok{]],}
    \CommentTok{\# incidenet energy }
    \StringTok{\textquotesingle{}E\textquotesingle{}}\NormalTok{: [}\FloatTok{0.0}\NormalTok{], }\CommentTok{\# Changed in loop}
    \CommentTok{\# cutoff energy {-} if E \textless{} Ec, particle stops}
    \StringTok{\textquotesingle{}Ec\textquotesingle{}}\NormalTok{: [helium[}\StringTok{"Ec"}\NormalTok{]],}
    \CommentTok{\# surface binding energy}
    \StringTok{\textquotesingle{}Es\textquotesingle{}}\NormalTok{: [helium[}\StringTok{"Es"}\NormalTok{]],}
    \CommentTok{\# initial position {-} if Es significant and E low, start (n)\^{}({-}1/3) above surface}
    \CommentTok{\# otherwise 0, 0, 0 is fine; most geometry modes have surface at x=0 with target x\textgreater{}0}
    \StringTok{\textquotesingle{}pos\textquotesingle{}}\NormalTok{: [[}\FloatTok{0.0}\NormalTok{, }\FloatTok{0.0}\NormalTok{, }\FloatTok{0.0}\NormalTok{]],}
    \CommentTok{\# initial direction unit vector; most geometry modes have x{-}axis into the surface}
    \StringTok{\textquotesingle{}dir\textquotesingle{}}\NormalTok{: [}
\NormalTok{        [}
\NormalTok{            np.cos(angle}\OperatorTok{*}\NormalTok{np.pi}\OperatorTok{/}\FloatTok{180.0}\NormalTok{),}
\NormalTok{            np.sin(angle}\OperatorTok{*}\NormalTok{np.pi}\OperatorTok{/}\FloatTok{180.0}\NormalTok{),}
            \FloatTok{0.0}
\NormalTok{        ]}
\NormalTok{        ],}
\NormalTok{    \}}

\CommentTok{\# Relevant constants}
\NormalTok{Z }\OperatorTok{=}\NormalTok{ helium[}\StringTok{"Z"}\NormalTok{]}
\NormalTok{M }\OperatorTok{=}\NormalTok{ helium[}\StringTok{"m"}\NormalTok{]}

\CommentTok{\# Loop over incident energies}
\ControlFlowTok{for}\NormalTok{ incident\_energy\_mev\_per\_amu, incident\_energy }\KeywordTok{in} \BuiltInTok{zip}\NormalTok{(helium\_energies, helium\_energies\_in):}
    \BuiltInTok{print}\NormalTok{(}\SpecialStringTok{f\textquotesingle{}Running simulation for incident energy: }\SpecialCharTok{\{}\NormalTok{incident\_energy\_mev\_per\_amu}\SpecialCharTok{\}}\SpecialStringTok{ MeV/amu\textquotesingle{}}\NormalTok{)}
\NormalTok{    particle\_parameters[}\StringTok{\textquotesingle{}E\textquotesingle{}}\NormalTok{] }\OperatorTok{=}\NormalTok{ [incident\_energy]}
\NormalTok{    input\_data }\OperatorTok{=}\NormalTok{ \{}
        \StringTok{\textquotesingle{}options\textquotesingle{}}\NormalTok{: options,}
        \StringTok{\textquotesingle{}material\_parameters\textquotesingle{}}\NormalTok{: material\_parameters,}
        \StringTok{\textquotesingle{}particle\_parameters\textquotesingle{}}\NormalTok{: particle\_parameters,}
        \StringTok{\textquotesingle{}geometry\_input\textquotesingle{}}\NormalTok{: geometry\_0D}
\NormalTok{    \}}

    \CommentTok{\# Attempt to cleanup line endings}
\NormalTok{    input\_string }\OperatorTok{=}\NormalTok{ dumps(input\_data).replace(}\StringTok{\textquotesingle{}}\CharTok{\textbackslash{}r}\StringTok{\textquotesingle{}}\NormalTok{, }\StringTok{\textquotesingle{}\textquotesingle{}}\NormalTok{)}
    \ControlFlowTok{with}  \BuiltInTok{open}\NormalTok{(}\StringTok{\textquotesingle{}examples/input\_file.toml\textquotesingle{}}\NormalTok{, }\StringTok{\textquotesingle{}w\textquotesingle{}}\NormalTok{) }\ImportTok{as}\NormalTok{ input\_file:}
\NormalTok{        input\_file.write(input\_string)}

    \ControlFlowTok{if}\NormalTok{ run\_sim:}
\NormalTok{        os.system(}\SpecialStringTok{f\textquotesingle{}cargo run {-}{-}release }\SpecialCharTok{\{}\NormalTok{mode}\SpecialCharTok{\}}\SpecialStringTok{ examples/input\_file.toml\textquotesingle{}}\NormalTok{)}

    \CommentTok{\# Read CSV in chunks to avoid memory issues}
\NormalTok{    loss\_df }\OperatorTok{=}\NormalTok{ dd.read\_csv(}\StringTok{\textquotesingle{}input\_fileenergy\_loss.output\textquotesingle{}}\NormalTok{,}
\NormalTok{                    header}\OperatorTok{=}\VariableTok{None}\NormalTok{,}
\NormalTok{                    dtype}\OperatorTok{=} \BuiltInTok{float}\NormalTok{,}
\NormalTok{                    blocksize}\OperatorTok{=}\StringTok{"64MB"}\NormalTok{).dropna()}
    
    \CommentTok{\# Process histogram in chunks without loading all data into memory}
\NormalTok{    depth\_col }\OperatorTok{=} \DecValTok{4}
\NormalTok{    energy\_cols }\OperatorTok{=}\NormalTok{ [}\DecValTok{2}\NormalTok{, }\DecValTok{3}\NormalTok{]}
    
    \CommentTok{\# Construct histogram for energy deposition function to measurement set by user}
\NormalTok{    bin\_edges }\OperatorTok{=}\NormalTok{ energy\_read\_range}
\NormalTok{    hist }\OperatorTok{=}\NormalTok{ np.zeros(}\BuiltInTok{len}\NormalTok{(bin\_edges) }\OperatorTok{{-}} \DecValTok{1}\NormalTok{)}
    
    \CommentTok{\# Process in chunks}
    \ControlFlowTok{for}\NormalTok{ partition }\KeywordTok{in}\NormalTok{ loss\_df.to\_delayed():}
\NormalTok{        chunk }\OperatorTok{=}\NormalTok{ partition.compute()}
        \ControlFlowTok{if} \BuiltInTok{len}\NormalTok{(chunk) }\OperatorTok{\textgreater{}} \DecValTok{0}\NormalTok{:}
\NormalTok{            depth }\OperatorTok{=}\NormalTok{ chunk.iloc[:, depth\_col].values}
\NormalTok{            energy }\OperatorTok{=}\NormalTok{ chunk.iloc[:, energy\_cols[}\DecValTok{0}\NormalTok{]].values }\OperatorTok{+}\NormalTok{ chunk.iloc[:, energy\_cols[}\DecValTok{1}\NormalTok{]].values}
\NormalTok{            chunk\_hist, \_ }\OperatorTok{=}\NormalTok{ np.histogram(depth, bins}\OperatorTok{=}\NormalTok{bin\_edges, weights}\OperatorTok{=}\NormalTok{energy)}
\NormalTok{            hist }\OperatorTok{+=}\NormalTok{ chunk\_hist}
    
\NormalTok{    loss }\OperatorTok{=} \VariableTok{None}  \CommentTok{\# Don\textquotesingle{}t need full array anymore}

    \ControlFlowTok{if}\NormalTok{ hist.}\BuiltInTok{sum}\NormalTok{() }\OperatorTok{==} \DecValTok{0}\NormalTok{:}
        \BuiltInTok{print}\NormalTok{(}\StringTok{\textquotesingle{}No energy loss data\textquotesingle{}}\NormalTok{)}
    \ControlFlowTok{else}\NormalTok{:}
        \CommentTok{\# Histogram energy loss data already computed above}
        \CommentTok{\# Total x coordinate range is 0 to 300000 A with 10000 bins}
\NormalTok{        widths }\OperatorTok{=}\NormalTok{ np.diff(bin\_edges)}
\NormalTok{        energy\_density }\OperatorTok{=}\NormalTok{ np.divide(}
\NormalTok{            hist,}
\NormalTok{            widths }\OperatorTok{*}\NormalTok{ number\_ions,}
\NormalTok{            out}\OperatorTok{=}\NormalTok{np.zeros\_like(hist, dtype}\OperatorTok{=}\BuiltInTok{float}\NormalTok{),}
\NormalTok{            where}\OperatorTok{=}\NormalTok{widths }\OperatorTok{!=} \DecValTok{0}
\NormalTok{        )}

\NormalTok{    total\_loss\_per\_ion }\OperatorTok{=}\NormalTok{ hist.}\BuiltInTok{sum}\NormalTok{() }\OperatorTok{/}\NormalTok{ number\_ions}

    \BuiltInTok{print}\NormalTok{(}\StringTok{\textquotesingle{}Total energy loss per ion:\textquotesingle{}}\NormalTok{, total\_loss\_per\_ion, }\StringTok{\textquotesingle{}eV\textquotesingle{}}\NormalTok{)}

    \CommentTok{\# Print energy per ion in particular range (e.g., 0 to 1000 A)}
\NormalTok{    range\_min }\OperatorTok{=} \FloatTok{0.0} \CommentTok{\# Depletion region}
\NormalTok{    range\_max }\OperatorTok{=} \FloatTok{35000.0} \CommentTok{\#  End of target}
\NormalTok{    mask }\OperatorTok{=}\NormalTok{ (bin\_edges[:}\OperatorTok{{-}}\DecValTok{1}\NormalTok{] }\OperatorTok{\textgreater{}=}\NormalTok{ range\_min) }\OperatorTok{\&}\NormalTok{ (bin\_edges[:}\OperatorTok{{-}}\DecValTok{1}\NormalTok{] }\OperatorTok{\textless{}}\NormalTok{ range\_max)}
\NormalTok{    energy\_in\_range }\OperatorTok{=}\NormalTok{ np.}\BuiltInTok{sum}\NormalTok{(energy\_density[mask] }\OperatorTok{*}\NormalTok{ widths[mask])}
\NormalTok{    percent\_loss\_in\_range }\OperatorTok{=}\NormalTok{ (energy\_in\_range }\OperatorTok{/}\NormalTok{ incident\_energy) }\OperatorTok{*} \FloatTok{100.0}
\NormalTok{    stopping\_power }\OperatorTok{=}\NormalTok{ energy\_in\_range }\OperatorTok{/}\NormalTok{ (range\_max }\OperatorTok{{-}}\NormalTok{ range\_min)}
    \BuiltInTok{print}\NormalTok{(}\SpecialStringTok{f\textquotesingle{}Energy loss per ion in range }\SpecialCharTok{\{}\NormalTok{range\_min}\SpecialCharTok{\}}\SpecialStringTok{ A to }\SpecialCharTok{\{}\NormalTok{range\_max}\SpecialCharTok{\}}\SpecialStringTok{ A: }\SpecialCharTok{\{}\NormalTok{energy\_in\_range}\SpecialCharTok{\}}\SpecialStringTok{ eV (}\SpecialCharTok{\{}\NormalTok{percent\_loss\_in\_range}\SpecialCharTok{:.2f\}}\SpecialStringTok{\%)\textquotesingle{}}\NormalTok{)}
    \BuiltInTok{print}\NormalTok{(}\SpecialStringTok{f\textquotesingle{}Stopping power in range }\SpecialCharTok{\{}\NormalTok{range\_min}\SpecialCharTok{\}}\SpecialStringTok{ A to }\SpecialCharTok{\{}\NormalTok{range\_max}\SpecialCharTok{\}}\SpecialStringTok{ A: }\SpecialCharTok{\{}\NormalTok{stopping\_power}\SpecialCharTok{\}}\SpecialStringTok{ eV/A/ion\textquotesingle{}}\NormalTok{)}
    \CommentTok{\# Convert to stopping power in KeV/um/Z**2 for comparison}
\NormalTok{    stopping\_power\_keV\_per\_um\_per\_Z2 }\OperatorTok{=}\NormalTok{ (stopping\_power }\OperatorTok{*} \FloatTok{1e{-}3}\NormalTok{) }\OperatorTok{/}\NormalTok{ (}\FloatTok{1e{-}4}\NormalTok{) }\OperatorTok{/}\NormalTok{ (Z}\OperatorTok{**}\DecValTok{2}\NormalTok{)}
    \CommentTok{\# Convert energy to MeV/amu for comparison}
\NormalTok{    incident\_energy\_MeV\_per\_amu }\OperatorTok{=}\NormalTok{ incident\_energy\_mev\_per\_amu}
\NormalTok{    stopping\_data.append(\{}
        \StringTok{\textquotesingle{}Incident Energy (MeV/amu)\textquotesingle{}}\NormalTok{: incident\_energy\_MeV\_per\_amu,}
        \StringTok{\textquotesingle{}Stopping Power (KeV/um/Z\^{}2)\textquotesingle{}}\NormalTok{: stopping\_power\_keV\_per\_um\_per\_Z2,}
        \StringTok{\textquotesingle{}Percent Energy Loss (\%)\textquotesingle{}}\NormalTok{: percent\_loss\_in\_range}
\NormalTok{    \})}

\CommentTok{\# Revert to output directory for output}
\NormalTok{os.chdir(output\_dir)}

\CommentTok{\# Write to file}
\NormalTok{stopping\_powers\_df }\OperatorTok{=}\NormalTok{ pd.DataFrame(stopping\_data)}
\NormalTok{stopping\_powers\_df.to\_csv(}\StringTok{\textquotesingle{}diamond\_he\_stopping\_powers.csv\textquotesingle{}}\NormalTok{, index}\OperatorTok{=}\VariableTok{False}\NormalTok{)}
\end{Highlighting}
\end{Shaded}

\begin{verbatim}
Running simulation for incident energy: 10.0 MeV/amu
\end{verbatim}

\begin{verbatim}
    Finished `release` profile [optimized] target(s) in 0.15s
     Running `target/release/RustBCA 0D examples/input_file.toml`
\end{verbatim}

\begin{verbatim}
Processing 100 ions...
Initializing with 10 threads...
Finished!
Total energy loss per ion: 1748734.449968571 eV
Energy loss per ion in range 0.0 A to 35000.0 A: 200815.1988926194 eV (0.50%)
Stopping power in range 0.0 A to 35000.0 A: 5.737577111217697 eV/A/ion
Running simulation for incident energy: 11.0 MeV/amu
\end{verbatim}

\begin{verbatim}
    Finished `release` profile [optimized] target(s) in 0.13s
     Running `target/release/RustBCA 0D examples/input_file.toml`
\end{verbatim}

\begin{verbatim}
Processing 100 ions...
Initializing with 10 threads...
Total energy loss per ion: 177510.81212419123 eV
Energy loss per ion in range 0.0 A to 35000.0 A: 20442.005919808376 eV (0.05%)
Stopping power in range 0.0 A to 35000.0 A: 0.584057311994525 eV/A/ion
Running simulation for incident energy: 12.0 MeV/amu
\end{verbatim}

\begin{verbatim}
    Finished `release` profile [optimized] target(s) in 0.13s
     Running `target/release/RustBCA 0D examples/input_file.toml`
\end{verbatim}

\begin{verbatim}
Processing 100 ions...
Initializing with 10 threads...
\end{verbatim}

\subsection{Lithium}\label{lithium}

\begin{Shaded}
\begin{Highlighting}[]
\CommentTok{\#\#\#\#\#\#\#\#\#\#\#\#\#\#\#\#\#\#\#\#\#\#\#\#\#\#\#\#\#\#\#\#\#\#\#\#\#\#\#\#\#\#\#\#\#}
\CommentTok{\#\# Simulation Options                      \#\#}
\CommentTok{\#\#\#{-}{-}{-}{-}{-}{-}{-}{-}{-}{-}{-}{-}{-}{-}{-}{-}{-}{-}{-}{-}{-}{-}{-}{-}{-}{-}{-}{-}{-}{-}{-}{-}{-}{-}{-}{-}{-}{-}{-}}\AlertTok{\#\#\#}
 \CommentTok{\# Lithium ion energies to simulate in MeV/amu}
\NormalTok{lithium\_energies }\OperatorTok{=}\NormalTok{ np.arange(}\FloatTok{20.0}\NormalTok{, }\FloatTok{31.0}\NormalTok{, }\FloatTok{1.0}\NormalTok{)}
 \CommentTok{\# Read range of energy deposition in Angstroms, and number of bins (e.g., numerical subdivisions of integral)}
\NormalTok{energy\_read\_range }\OperatorTok{=}\NormalTok{ np.linspace(}\FloatTok{0.0}\NormalTok{, }\FloatTok{3e5}\NormalTok{, }\DecValTok{10001}\NormalTok{)}
\CommentTok{\#\#\#{-}{-}{-}{-}{-}{-}{-}{-}{-}{-}{-}{-}{-}{-}{-}{-}{-}{-}{-}{-}{-}{-}{-}{-}{-}{-}{-}{-}{-}{-}{-}{-}{-}{-}{-}{-}{-}{-}{-}}\AlertTok{\#\#\#}
\CommentTok{\#\# End Simulation Options                  \#\#}
\CommentTok{\#\#\#\#\#\#\#\#\#\#\#\#\#\#\#\#\#\#\#\#\#\#\#\#\#\#\#\#\#\#\#\#\#\#\#\#\#\#\#\#\#\#\#\#\#}

\CommentTok{\# Convert lithium energies to raw energies in eV for input file}
\NormalTok{lithium\_energies\_in }\OperatorTok{=}\NormalTok{ lithium\_energies }\OperatorTok{*} \FloatTok{1e6} \OperatorTok{*}\NormalTok{ lithium[}\StringTok{"m"}\NormalTok{]}

\NormalTok{os.chdir(rustbca\_dir)}
\NormalTok{stopping\_data }\OperatorTok{=}\NormalTok{ [] }\CommentTok{\# Collect stopping powers here in list}

\NormalTok{lithium }\OperatorTok{=}\NormalTok{ m.lithium}

\NormalTok{particle\_parameters }\OperatorTok{=}\NormalTok{ \{}
    \StringTok{\textquotesingle{}length\_unit\textquotesingle{}}\NormalTok{: }\StringTok{\textquotesingle{}ANGSTROM\textquotesingle{}}\NormalTok{,}
    \StringTok{\textquotesingle{}energy\_unit\textquotesingle{}}\NormalTok{: }\StringTok{\textquotesingle{}EV\textquotesingle{}}\NormalTok{,}
    \StringTok{\textquotesingle{}mass\_unit\textquotesingle{}}\NormalTok{: }\StringTok{\textquotesingle{}AMU\textquotesingle{}}\NormalTok{,}
    \CommentTok{\# number of computational ions of this species to run at this energy}
    \StringTok{\textquotesingle{}N\textquotesingle{}}\NormalTok{: [number\_ions],}
    \CommentTok{\# atomic mass}
    \StringTok{\textquotesingle{}m\textquotesingle{}}\NormalTok{: [lithium[}\StringTok{"m"}\NormalTok{]],}
    \CommentTok{\# atomic number}
    \StringTok{\textquotesingle{}Z\textquotesingle{}}\NormalTok{: [lithium[}\StringTok{"Z"}\NormalTok{]],}
    \CommentTok{\# incidenet energy }
    \StringTok{\textquotesingle{}E\textquotesingle{}}\NormalTok{: [}\FloatTok{0.0}\NormalTok{], }\CommentTok{\# Changed in loop}
    \CommentTok{\# cutoff energy {-} if E \textless{} Ec, particle stops}
    \StringTok{\textquotesingle{}Ec\textquotesingle{}}\NormalTok{: [lithium[}\StringTok{"Ec"}\NormalTok{]],}
    \CommentTok{\# surface binding energy}
    \StringTok{\textquotesingle{}Es\textquotesingle{}}\NormalTok{: [lithium[}\StringTok{"Es"}\NormalTok{]],}
    \CommentTok{\# initial position {-} if Es significant and E low, start (n)\^{}({-}1/3) above surface}
    \CommentTok{\# otherwise 0, 0, 0 is fine; most geometry modes have surface at x=0 with target x\textgreater{}0}
    \StringTok{\textquotesingle{}pos\textquotesingle{}}\NormalTok{: [[}\FloatTok{0.0}\NormalTok{, }\FloatTok{0.0}\NormalTok{, }\FloatTok{0.0}\NormalTok{]],}
    \CommentTok{\# initial direction unit vector; most geometry modes have x{-}axis into the surface}
    \StringTok{\textquotesingle{}dir\textquotesingle{}}\NormalTok{: [}
\NormalTok{        [}
\NormalTok{            np.cos(angle}\OperatorTok{*}\NormalTok{np.pi}\OperatorTok{/}\FloatTok{180.0}\NormalTok{),}
\NormalTok{            np.sin(angle}\OperatorTok{*}\NormalTok{np.pi}\OperatorTok{/}\FloatTok{180.0}\NormalTok{),}
            \FloatTok{0.0}
\NormalTok{        ]}
\NormalTok{        ],}
\NormalTok{    \}}

\CommentTok{\# Relevant constants}
\NormalTok{Z }\OperatorTok{=}\NormalTok{ lithium[}\StringTok{"Z"}\NormalTok{]}
\NormalTok{M }\OperatorTok{=}\NormalTok{ lithium[}\StringTok{"m"}\NormalTok{]}

\CommentTok{\# Loop over incident energies}
\ControlFlowTok{for}\NormalTok{ incident\_energy\_mev\_per\_amu, incident\_energy }\KeywordTok{in} \BuiltInTok{zip}\NormalTok{(lithium\_energies, lithium\_energies\_in):}
    \BuiltInTok{print}\NormalTok{(}\SpecialStringTok{f\textquotesingle{}Running simulation for incident energy: }\SpecialCharTok{\{}\NormalTok{incident\_energy\_mev\_per\_amu}\SpecialCharTok{\}}\SpecialStringTok{ MeV/amu\textquotesingle{}}\NormalTok{)}
\NormalTok{    particle\_parameters[}\StringTok{\textquotesingle{}E\textquotesingle{}}\NormalTok{] }\OperatorTok{=}\NormalTok{ [incident\_energy]}
\NormalTok{    input\_data }\OperatorTok{=}\NormalTok{ \{ }
        \StringTok{\textquotesingle{}options\textquotesingle{}}\NormalTok{: options,}
        \StringTok{\textquotesingle{}material\_parameters\textquotesingle{}}\NormalTok{: material\_parameters,}
        \StringTok{\textquotesingle{}particle\_parameters\textquotesingle{}}\NormalTok{: particle\_parameters,}
        \StringTok{\textquotesingle{}geometry\_input\textquotesingle{}}\NormalTok{: geometry\_0D}
\NormalTok{    \}}

    \CommentTok{\# Attempt to cleanup line endings}
\NormalTok{    input\_string }\OperatorTok{=}\NormalTok{ dumps(input\_data).replace(}\StringTok{\textquotesingle{}}\CharTok{\textbackslash{}r}\StringTok{\textquotesingle{}}\NormalTok{, }\StringTok{\textquotesingle{}\textquotesingle{}}\NormalTok{)}
    \ControlFlowTok{with}  \BuiltInTok{open}\NormalTok{(}\StringTok{\textquotesingle{}examples/input\_file.toml\textquotesingle{}}\NormalTok{, }\StringTok{\textquotesingle{}w\textquotesingle{}}\NormalTok{) }\ImportTok{as}\NormalTok{ input\_file:}
\NormalTok{        input\_file.write(input\_string)}

    \ControlFlowTok{if}\NormalTok{ run\_sim:}
\NormalTok{        os.system(}\SpecialStringTok{f\textquotesingle{}cargo run {-}{-}release }\SpecialCharTok{\{}\NormalTok{mode}\SpecialCharTok{\}}\SpecialStringTok{ examples/input\_file.toml\textquotesingle{}}\NormalTok{)}

    \CommentTok{\# Read CSV in chunks to avoid memory issues}
\NormalTok{    loss\_df }\OperatorTok{=}\NormalTok{ dd.read\_csv(}\StringTok{\textquotesingle{}input\_fileenergy\_loss.output\textquotesingle{}}\NormalTok{,}
\NormalTok{                    header}\OperatorTok{=}\VariableTok{None}\NormalTok{,}
\NormalTok{                    dtype}\OperatorTok{=} \BuiltInTok{float}\NormalTok{,}
\NormalTok{                    blocksize}\OperatorTok{=}\StringTok{"64MB"}\NormalTok{).dropna()}
    
    \CommentTok{\# Process histogram in chunks without loading all data into memory}
\NormalTok{    depth\_col }\OperatorTok{=} \DecValTok{4}
\NormalTok{    energy\_cols }\OperatorTok{=}\NormalTok{ [}\DecValTok{2}\NormalTok{, }\DecValTok{3}\NormalTok{]}
    
    \CommentTok{\# Construct histogram for energy deposition function to measurement set by user}
\NormalTok{    bin\_edges }\OperatorTok{=}\NormalTok{ energy\_read\_range}
\NormalTok{    hist }\OperatorTok{=}\NormalTok{ np.zeros(}\BuiltInTok{len}\NormalTok{(bin\_edges) }\OperatorTok{{-}} \DecValTok{1}\NormalTok{)}
    
    \CommentTok{\# Process in chunks}
    \ControlFlowTok{for}\NormalTok{ partition }\KeywordTok{in}\NormalTok{ loss\_df.to\_delayed():}
\NormalTok{        chunk }\OperatorTok{=}\NormalTok{ partition.compute()}
        \ControlFlowTok{if} \BuiltInTok{len}\NormalTok{(chunk) }\OperatorTok{\textgreater{}} \DecValTok{0}\NormalTok{:}
\NormalTok{            depth }\OperatorTok{=}\NormalTok{ chunk.iloc[:, depth\_col].values}
\NormalTok{            energy }\OperatorTok{=}\NormalTok{ chunk.iloc[:, energy\_cols[}\DecValTok{0}\NormalTok{]].values }\OperatorTok{+}\NormalTok{ chunk.iloc[:, energy\_cols[}\DecValTok{1}\NormalTok{]].values}
\NormalTok{            chunk\_hist, \_ }\OperatorTok{=}\NormalTok{ np.histogram(depth, bins}\OperatorTok{=}\NormalTok{bin\_edges, weights}\OperatorTok{=}\NormalTok{energy)}
\NormalTok{            hist }\OperatorTok{+=}\NormalTok{ chunk\_hist}
    
\NormalTok{    loss }\OperatorTok{=} \VariableTok{None}  \CommentTok{\# Don\textquotesingle{}t need full array anymore}

    \ControlFlowTok{if}\NormalTok{ hist.}\BuiltInTok{sum}\NormalTok{() }\OperatorTok{==} \DecValTok{0}\NormalTok{:}
        \BuiltInTok{print}\NormalTok{(}\StringTok{\textquotesingle{}No energy loss data\textquotesingle{}}\NormalTok{)}
    \ControlFlowTok{else}\NormalTok{:}
        \CommentTok{\# Histogram energy loss data already computed above}
        \CommentTok{\# Total x coordinate range is 0 to 300000 A with 10000 bins}
\NormalTok{        widths }\OperatorTok{=}\NormalTok{ np.diff(bin\_edges)}
\NormalTok{        energy\_density }\OperatorTok{=}\NormalTok{ np.divide(}
\NormalTok{            hist,}
\NormalTok{            widths }\OperatorTok{*}\NormalTok{ number\_ions,}
\NormalTok{            out}\OperatorTok{=}\NormalTok{np.zeros\_like(hist, dtype}\OperatorTok{=}\BuiltInTok{float}\NormalTok{),}
\NormalTok{            where}\OperatorTok{=}\NormalTok{widths }\OperatorTok{!=} \DecValTok{0}
\NormalTok{        )}

\NormalTok{    total\_loss\_per\_ion }\OperatorTok{=}\NormalTok{ hist.}\BuiltInTok{sum}\NormalTok{() }\OperatorTok{/}\NormalTok{ number\_ions}

    \BuiltInTok{print}\NormalTok{(}\StringTok{\textquotesingle{}Total energy loss per ion:\textquotesingle{}}\NormalTok{, total\_loss\_per\_ion, }\StringTok{\textquotesingle{}eV\textquotesingle{}}\NormalTok{)}

    \CommentTok{\# Print energy per ion in particular range (e.g., 0 to 1000 A)}
\NormalTok{    range\_min }\OperatorTok{=} \FloatTok{0.0} \CommentTok{\# Depletion region}
\NormalTok{    range\_max }\OperatorTok{=} \FloatTok{35000.0} \CommentTok{\#  End of target}
\NormalTok{    mask }\OperatorTok{=}\NormalTok{ (bin\_edges[:}\OperatorTok{{-}}\DecValTok{1}\NormalTok{] }\OperatorTok{\textgreater{}=}\NormalTok{ range\_min) }\OperatorTok{\&}\NormalTok{ (bin\_edges[:}\OperatorTok{{-}}\DecValTok{1}\NormalTok{] }\OperatorTok{\textless{}}\NormalTok{ range\_max)}
\NormalTok{    energy\_in\_range }\OperatorTok{=}\NormalTok{ np.}\BuiltInTok{sum}\NormalTok{(energy\_density[mask] }\OperatorTok{*}\NormalTok{ widths[mask])}
\NormalTok{    percent\_loss\_in\_range }\OperatorTok{=}\NormalTok{ (energy\_in\_range }\OperatorTok{/}\NormalTok{ incident\_energy) }\OperatorTok{*} \FloatTok{100.0}
\NormalTok{    stopping\_power }\OperatorTok{=}\NormalTok{ energy\_in\_range }\OperatorTok{/}\NormalTok{ (range\_max }\OperatorTok{{-}}\NormalTok{ range\_min)}
    \BuiltInTok{print}\NormalTok{(}\SpecialStringTok{f\textquotesingle{}Energy loss per ion in range }\SpecialCharTok{\{}\NormalTok{range\_min}\SpecialCharTok{\}}\SpecialStringTok{ A to }\SpecialCharTok{\{}\NormalTok{range\_max}\SpecialCharTok{\}}\SpecialStringTok{ A: }\SpecialCharTok{\{}\NormalTok{energy\_in\_range}\SpecialCharTok{\}}\SpecialStringTok{ eV (}\SpecialCharTok{\{}\NormalTok{percent\_loss\_in\_range}\SpecialCharTok{:.2f\}}\SpecialStringTok{\%)\textquotesingle{}}\NormalTok{)}
    \BuiltInTok{print}\NormalTok{(}\SpecialStringTok{f\textquotesingle{}Stopping power in range }\SpecialCharTok{\{}\NormalTok{range\_min}\SpecialCharTok{\}}\SpecialStringTok{ A to }\SpecialCharTok{\{}\NormalTok{range\_max}\SpecialCharTok{\}}\SpecialStringTok{ A: }\SpecialCharTok{\{}\NormalTok{stopping\_power}\SpecialCharTok{\}}\SpecialStringTok{ eV/A/ion\textquotesingle{}}\NormalTok{)}
    \CommentTok{\# Convert to stopping power in KeV/um/Z**2 for comparison}
\NormalTok{    stopping\_power\_keV\_per\_um\_per\_Z2 }\OperatorTok{=}\NormalTok{ (stopping\_power }\OperatorTok{*} \FloatTok{1e{-}3}\NormalTok{) }\OperatorTok{/}\NormalTok{ (}\FloatTok{1e{-}4}\NormalTok{) }\OperatorTok{/}\NormalTok{ (Z}\OperatorTok{**}\DecValTok{2}\NormalTok{)}
    \CommentTok{\# Convert energy to MeV/amu for comparison}
\NormalTok{    incident\_energy\_MeV\_per\_amu }\OperatorTok{=}\NormalTok{ incident\_energy\_mev\_per\_amu}
\NormalTok{    stopping\_data.append(\{}
        \StringTok{\textquotesingle{}Incident Energy (MeV/amu)\textquotesingle{}}\NormalTok{: incident\_energy\_MeV\_per\_amu,}
        \StringTok{\textquotesingle{}Stopping Power (KeV/um/Z\^{}2)\textquotesingle{}}\NormalTok{: stopping\_power\_keV\_per\_um\_per\_Z2,}
        \StringTok{\textquotesingle{}Percent Energy Loss (\%)\textquotesingle{}}\NormalTok{: percent\_loss\_in\_range}
\NormalTok{    \})}

\CommentTok{\# Revert to output directory for output}
\NormalTok{os.chdir(output\_dir)}

\CommentTok{\# Write to file}
\NormalTok{stopping\_powers\_df }\OperatorTok{=}\NormalTok{ pd.DataFrame(stopping\_data)}
\NormalTok{stopping\_powers\_df.to\_csv(}\StringTok{\textquotesingle{}diamond\_li\_stopping\_powers.csv\textquotesingle{}}\NormalTok{, index}\OperatorTok{=}\VariableTok{False}\NormalTok{)}
\end{Highlighting}
\end{Shaded}

\begin{verbatim}
Running simulation for incident energy: 20000000.0 eV
\end{verbatim}

\begin{verbatim}
    Finished `release` profile [optimized] target(s) in 0.13s
     Running `target/release/RustBCA 0D examples/input_file.toml`
\end{verbatim}

\begin{verbatim}
Processing 100 ions...
Initializing with 10 threads...
Finished!
Total energy loss per ion: 12806589.509847958 eV
Energy loss per ion in range 0.0 A to 35000.0 A: 1171237.4879509213 eV (5.86%)
Stopping power in range 0.0 A to 35000.0 A: 33.46392822716918 eV/A/ion
Running simulation for incident energy: 21000000.0 eV
\end{verbatim}

\begin{verbatim}
    Finished `release` profile [optimized] target(s) in 0.11s
     Running `target/release/RustBCA 0D examples/input_file.toml`
\end{verbatim}

\begin{verbatim}
Processing 100 ions...
Initializing with 10 threads...
Finished!
Total energy loss per ion: 12000841.2930202 eV
Energy loss per ion in range 0.0 A to 35000.0 A: 1130011.998248974 eV (5.38%)
Stopping power in range 0.0 A to 35000.0 A: 32.28605709282783 eV/A/ion
Running simulation for incident energy: 22000000.0 eV
\end{verbatim}

\begin{verbatim}
    Finished `release` profile [optimized] target(s) in 0.13s
     Running `target/release/RustBCA 0D examples/input_file.toml`
\end{verbatim}

\begin{verbatim}
Processing 100 ions...
Initializing with 10 threads...
Finished!
Total energy loss per ion: 11337089.034799255 eV
Energy loss per ion in range 0.0 A to 35000.0 A: 1091276.904355215 eV (4.96%)
Stopping power in range 0.0 A to 35000.0 A: 31.179340124434713 eV/A/ion
Running simulation for incident energy: 23000000.0 eV
\end{verbatim}

\begin{verbatim}
    Finished `release` profile [optimized] target(s) in 0.12s
     Running `target/release/RustBCA 0D examples/input_file.toml`
\end{verbatim}

\begin{verbatim}
Processing 100 ions...
Initializing with 10 threads...
Finished!
Total energy loss per ion: 10749828.335853502 eV
Energy loss per ion in range 0.0 A to 35000.0 A: 1055835.1130366486 eV (4.59%)
Stopping power in range 0.0 A to 35000.0 A: 30.166717515332817 eV/A/ion
Running simulation for incident energy: 24000000.0 eV
\end{verbatim}

\begin{verbatim}
    Finished `release` profile [optimized] target(s) in 0.12s
     Running `target/release/RustBCA 0D examples/input_file.toml`
\end{verbatim}

\begin{verbatim}
Processing 100 ions...
Initializing with 10 threads...
Finished!
Total energy loss per ion: 10248948.316350447 eV
Energy loss per ion in range 0.0 A to 35000.0 A: 1023186.356528989 eV (4.26%)
Stopping power in range 0.0 A to 35000.0 A: 29.23389590082826 eV/A/ion
Running simulation for incident energy: 25000000.0 eV
\end{verbatim}

\begin{verbatim}
    Finished `release` profile [optimized] target(s) in 0.13s
     Running `target/release/RustBCA 0D examples/input_file.toml`
\end{verbatim}

\begin{verbatim}
Processing 100 ions...
Initializing with 10 threads...
Finished!
Total energy loss per ion: 9805308.20336172 eV
Energy loss per ion in range 0.0 A to 35000.0 A: 991929.2552379267 eV (3.97%)
Stopping power in range 0.0 A to 35000.0 A: 28.340835863940764 eV/A/ion
Running simulation for incident energy: 26000000.0 eV
\end{verbatim}

\begin{verbatim}
    Finished `release` profile [optimized] target(s) in 0.14s
     Running `target/release/RustBCA 0D examples/input_file.toml`
\end{verbatim}

\begin{verbatim}
Processing 100 ions...
Initializing with 10 threads...
Finished!
Total energy loss per ion: 9408159.943932831 eV
Energy loss per ion in range 0.0 A to 35000.0 A: 962580.2823142118 eV (3.70%)
Stopping power in range 0.0 A to 35000.0 A: 27.502293780406053 eV/A/ion
Running simulation for incident energy: 27000000.0 eV
\end{verbatim}

\begin{verbatim}
    Finished `release` profile [optimized] target(s) in 0.14s
     Running `target/release/RustBCA 0D examples/input_file.toml`
\end{verbatim}

\begin{verbatim}
Processing 100 ions...
Initializing with 10 threads...
Finished!
Total energy loss per ion: 9052151.536107764 eV
Energy loss per ion in range 0.0 A to 35000.0 A: 935436.7108079125 eV (3.46%)
Stopping power in range 0.0 A to 35000.0 A: 26.72676316594036 eV/A/ion
Running simulation for incident energy: 28000000.0 eV
\end{verbatim}

\begin{verbatim}
    Finished `release` profile [optimized] target(s) in 0.14s
     Running `target/release/RustBCA 0D examples/input_file.toml`
\end{verbatim}

\begin{verbatim}
Processing 100 ions...
Initializing with 10 threads...
Finished!
Total energy loss per ion: 8730491.010769421 eV
Energy loss per ion in range 0.0 A to 35000.0 A: 909972.4461867402 eV (3.25%)
Stopping power in range 0.0 A to 35000.0 A: 25.999212748192576 eV/A/ion
Running simulation for incident energy: 29000000.0 eV
\end{verbatim}

\begin{verbatim}
    Finished `release` profile [optimized] target(s) in 0.13s
     Running `target/release/RustBCA 0D examples/input_file.toml`
\end{verbatim}

\begin{verbatim}
Processing 100 ions...
Initializing with 10 threads...
Finished!
Total energy loss per ion: 8431948.056560442 eV
Energy loss per ion in range 0.0 A to 35000.0 A: 886240.3693104261 eV (3.06%)
Stopping power in range 0.0 A to 35000.0 A: 25.321153408869318 eV/A/ion
Running simulation for incident energy: 30000000.0 eV
\end{verbatim}

\begin{verbatim}
    Finished `release` profile [optimized] target(s) in 0.13s
     Running `target/release/RustBCA 0D examples/input_file.toml`
\end{verbatim}

\begin{verbatim}
Processing 100 ions...
Initializing with 10 threads...
Finished!
Total energy loss per ion: 8159148.033519524 eV
Energy loss per ion in range 0.0 A to 35000.0 A: 863438.0915517404 eV (2.88%)
Stopping power in range 0.0 A to 35000.0 A: 24.669659758621155 eV/A/ion
\end{verbatim}

\section{Plots}\label{plots}

\begin{Shaded}
\begin{Highlighting}[]
\CommentTok{\# Now, we have three CSV files with stopping powers for H, He, and Li ions in diamond. We can read these files and plot the stopping power as a function of incident energy for each ion species.}
\ImportTok{import}\NormalTok{ pandas }\ImportTok{as}\NormalTok{ pd}
\ImportTok{import}\NormalTok{ matplotlib.pyplot }\ImportTok{as}\NormalTok{ plt}

\CommentTok{\#\#\# Visual Settings }\AlertTok{\#\#\#}
\CommentTok{\# Set the font family to \textquotesingle{}serif\textquotesingle{} and the serif font to \textquotesingle{}cmr10\textquotesingle{} (Computer Modern Roman)}
\NormalTok{plt.rcParams[}\StringTok{\textquotesingle{}font.family\textquotesingle{}}\NormalTok{] }\OperatorTok{=} \StringTok{\textquotesingle{}serif\textquotesingle{}}
\NormalTok{plt.rcParams[}\StringTok{\textquotesingle{}font.serif\textquotesingle{}}\NormalTok{] }\OperatorTok{=}\NormalTok{ [}\StringTok{\textquotesingle{}cmr10\textquotesingle{}}\NormalTok{]}
\NormalTok{plt.rcParams[}\StringTok{\textquotesingle{}font.size\textquotesingle{}}\NormalTok{] }\OperatorTok{=} \DecValTok{12}


\CommentTok{\# Change to data directory}
\NormalTok{os.chdir(output\_dir)}
\CommentTok{\# Read the CSV files}
\NormalTok{h\_data }\OperatorTok{=}\NormalTok{ pd.read\_csv(}\StringTok{\textquotesingle{}diamond\_h\_stopping\_powers.csv\textquotesingle{}}\NormalTok{)}
\NormalTok{he\_data }\OperatorTok{=}\NormalTok{ pd.read\_csv(}\StringTok{\textquotesingle{}diamond\_he\_stopping\_powers.csv\textquotesingle{}}\NormalTok{)}
\NormalTok{li\_data }\OperatorTok{=}\NormalTok{ pd.read\_csv(}\StringTok{\textquotesingle{}diamond\_li\_stopping\_powers.csv\textquotesingle{}}\NormalTok{)}
\CommentTok{\# Read experimental data from literature}
\NormalTok{experimental\_data }\OperatorTok{=}\NormalTok{ pd.read\_csv(}\StringTok{\textquotesingle{}experimental\_stopping\_powers.csv\textquotesingle{}}\NormalTok{)  }\CommentTok{\# Replace with actual file path}
\CommentTok{\# Separate species in experimental data}
\NormalTok{experimental\_h }\OperatorTok{=}\NormalTok{ experimental\_data[experimental\_data[}\StringTok{\textquotesingle{}Species\textquotesingle{}}\NormalTok{] }\OperatorTok{==} \StringTok{\textquotesingle{}H\textquotesingle{}}\NormalTok{]}
\NormalTok{experimental\_he }\OperatorTok{=}\NormalTok{ experimental\_data[experimental\_data[}\StringTok{\textquotesingle{}Species\textquotesingle{}}\NormalTok{] }\OperatorTok{==} \StringTok{\textquotesingle{}He\textquotesingle{}}\NormalTok{]}
\NormalTok{experimental\_li }\OperatorTok{=}\NormalTok{ experimental\_data[experimental\_data[}\StringTok{\textquotesingle{}Species\textquotesingle{}}\NormalTok{] }\OperatorTok{==} \StringTok{\textquotesingle{}Li\textquotesingle{}}\NormalTok{]}
\CommentTok{\# Plotting}
\NormalTok{plt.scatter(h\_data[}\StringTok{\textquotesingle{}Incident Energy (MeV/amu)\textquotesingle{}}\NormalTok{], h\_data[}\StringTok{\textquotesingle{}Stopping Power (KeV/um/Z\^{}2)\textquotesingle{}}\NormalTok{], label}\OperatorTok{=}\StringTok{\textquotesingle{}H\textquotesingle{}}\NormalTok{, marker}\OperatorTok{=}\StringTok{\textquotesingle{}o\textquotesingle{}}\NormalTok{, color }\OperatorTok{=} \StringTok{\textquotesingle{}black\textquotesingle{}}\NormalTok{) }\CommentTok{\# Change shapes for better visibility}
\CommentTok{\# Imagine what it looks like in black and white}
\NormalTok{plt.scatter(he\_data[}\StringTok{\textquotesingle{}Incident Energy (MeV/amu)\textquotesingle{}}\NormalTok{], he\_data[}\StringTok{\textquotesingle{}Stopping Power (KeV/um/Z\^{}2)\textquotesingle{}}\NormalTok{], label}\OperatorTok{=}\StringTok{\textquotesingle{}He\textquotesingle{}}\NormalTok{, marker}\OperatorTok{=}\StringTok{\textquotesingle{}v\textquotesingle{}}\NormalTok{, color }\OperatorTok{=} \StringTok{\textquotesingle{}black\textquotesingle{}}\NormalTok{)}
\NormalTok{plt.scatter(li\_data[}\StringTok{\textquotesingle{}Incident Energy (MeV/amu)\textquotesingle{}}\NormalTok{], li\_data[}\StringTok{\textquotesingle{}Stopping Power (KeV/um/Z\^{}2)\textquotesingle{}}\NormalTok{], label}\OperatorTok{=}\StringTok{\textquotesingle{}Li\textquotesingle{}}\NormalTok{, marker}\OperatorTok{=}\StringTok{\textquotesingle{}+\textquotesingle{}}\NormalTok{, color }\OperatorTok{=} \StringTok{\textquotesingle{}darkgray\textquotesingle{}}\NormalTok{)}
\NormalTok{plt.scatter(experimental\_h[}\StringTok{\textquotesingle{}Energy (MeV/amu)\textquotesingle{}}\NormalTok{], experimental\_h[}\StringTok{\textquotesingle{}Stopping Power (keV/um)\textquotesingle{}}\NormalTok{], label}\OperatorTok{=}\StringTok{\textquotesingle{}H (Experimental)\textquotesingle{}}\NormalTok{, marker}\OperatorTok{=}\StringTok{\textquotesingle{}1\textquotesingle{}}\NormalTok{, color }\OperatorTok{=} \StringTok{\textquotesingle{}gray\textquotesingle{}}\NormalTok{)}
\NormalTok{plt.scatter(experimental\_he[}\StringTok{\textquotesingle{}Energy (MeV/amu)\textquotesingle{}}\NormalTok{], experimental\_he[}\StringTok{\textquotesingle{}Stopping Power (keV/um)\textquotesingle{}}\NormalTok{], label}\OperatorTok{=}\StringTok{\textquotesingle{}He (Experimental)\textquotesingle{}}\NormalTok{, marker}\OperatorTok{=}\StringTok{\textquotesingle{}s\textquotesingle{}}\NormalTok{, color }\OperatorTok{=} \StringTok{\textquotesingle{}lightgray\textquotesingle{}}\NormalTok{)}
\NormalTok{plt.scatter(experimental\_li[}\StringTok{\textquotesingle{}Energy (MeV/amu)\textquotesingle{}}\NormalTok{], experimental\_li[}\StringTok{\textquotesingle{}Stopping Power (keV/um)\textquotesingle{}}\NormalTok{], label}\OperatorTok{=}\StringTok{\textquotesingle{}Li (Experimental)\textquotesingle{}}\NormalTok{, marker}\OperatorTok{=}\StringTok{\textquotesingle{}x\textquotesingle{}}\NormalTok{, color }\OperatorTok{=} \StringTok{\textquotesingle{}lightgray\textquotesingle{}}\NormalTok{)}
\NormalTok{plt.xlabel(}\StringTok{\textquotesingle{}Incident Energy (MeV/amu)\textquotesingle{}}\NormalTok{)}
\NormalTok{plt.ylabel(}\StringTok{\textquotesingle{}Stopping Power (KeV/$(}\ErrorTok{\textbackslash{}}\StringTok{mu m }\ErrorTok{\textbackslash{}}\StringTok{cdot Z\^{}2)$)\textquotesingle{}}\NormalTok{)}
\NormalTok{plt.title(}\StringTok{\textquotesingle{}Stopping Power of H, He, and Li Ions in Diamond\textquotesingle{}}\NormalTok{)}
\NormalTok{plt.loglog()  }\CommentTok{\# Use logarithmic scale for better visibility}
\CommentTok{\# Plot the Bethe stopping power using our previously defined function S(epsilon), on the same graph for comparison}
\NormalTok{incident\_energies }\OperatorTok{=}\NormalTok{ np.linspace(np.}\BuiltInTok{min}\NormalTok{(h\_data[}\StringTok{\textquotesingle{}Incident Energy (MeV/amu)\textquotesingle{}}\NormalTok{]), np.}\BuiltInTok{max}\NormalTok{(he\_data[}\StringTok{\textquotesingle{}Incident Energy (MeV/amu)\textquotesingle{}}\NormalTok{]), }\DecValTok{100}\NormalTok{)}
\NormalTok{function\_data }\OperatorTok{=}\NormalTok{ S(incident\_energies)}
\NormalTok{plt.plot(incident\_energies, function\_data, label}\OperatorTok{=}\StringTok{\textquotesingle{}Bethe Stopping Power\textquotesingle{}}\NormalTok{, linestyle}\OperatorTok{=}\StringTok{\textquotesingle{}{-}{-}\textquotesingle{}}\NormalTok{, color }\OperatorTok{=} \StringTok{\textquotesingle{}black\textquotesingle{}}\NormalTok{)}
\NormalTok{plt.legend()}
\NormalTok{plt.savefig(}\StringTok{\textquotesingle{}diamond\_stopping\_powers.png\textquotesingle{}}\NormalTok{)}
\NormalTok{plt.show()}
\end{Highlighting}
\end{Shaded}

\pandocbounded{\includegraphics[keepaspectratio]{diamond_compare_files/figure-pdf/cell-8-output-1.png}}




\end{document}
